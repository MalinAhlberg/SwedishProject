\documentclass[submission]{eptcs} %rätt lincens?
\providecommand{\event}{LREC2012}
\usepackage[utf8]{inputenc}
\usepackage{graphicx}

\title{First results of developing a Swedish parser based on a wide-coverage grammar\\
        eller Combining resources into a Swedish parser
        eller Bootstrapping a Swedish parser \\
        eller A grammar-based Swedish parser from already existing resources}
\author{Malin Ahlberg \& Ramona Enache
%\and
%Co Author \qquad\qquad 
\institute{Department of Computer Science \& Engineering, Gothenburg University, Sweden}
}
\def\titlerunning{First results of developing a grammar based Swedish parser}
\def\authorrunning{M. Ahlberg}
\begin{document}
\maketitle

\begin{abstract}
This paper describes on going work of a rule-based, open-source
parser for Swedish. The central component will be a wide-coverage grammar
implemented in the formalism Grammatical Framework (GF), from which we 
derive?? a deep parser.
In addition to GF, we use two other main resources: the Swedish
treebank Talbanken and the electronic lexicon SALDO.
The resulting parser will 
will give a full syntactic analysis of the input sentences.
Since all the parser will be freely available, 
% ha med detta?
it will be highly reusable and chosen parts of the the grammar can be used in
applications dealing with
%controlled or free
natural language. GF also provides libraries for compiling
grammars to a number of programming languages.
% allow easy interaction with other programs and grammars
%>> something very commercial, interesting and outstanding here please!

\end{abstract}

\section{Introduction}
Swedish is a North-Germanic language spoken by approximately 10 million people.
It is a SVO language, and although the word order is relatively strict,
inverted order is commonly used
in subordinate clauses, to indicate questions or to emphasize different parts of the sentence. \\
Our goal is to implement a wide coverage grammar and parser for Swedish
using the grammar formalism GF. 
% better 
We implement a Swedish grammar and use GF to compile it to a parser.
From this, we aim to implement a robust parser. 
We thereby investigate how GF
can be used for open-domain parsing.
The parser will be evaluated both by experts and by comparing the results of
parsing sentences in an commonly used Swedish treebank, Talbanken.

This paper will briefly introduce GF, Talbanken and SALDO in Section \ref{sec:background}
and describe our current work in Section \ref{sec:progress}.
%We aim to implement % The goal is to be able to 
%a parser for open domain language and 
%This paper describes our work and results 
%so far as well as the directions of future work.

\section{Background}
\label{sec:background}
\subsection{Grammatical Framework}
\label{sec:gf}

%städa upp detta stycke!!

Grammatical Framework\cite{ranta-2011} (GF) is a grammar formalism based on functional
programming, designed for multilingual applications.

The key idea is to divide the grammar into abstract and concrete parts. %% parts?
The abstract grammar gives a logical representation of the semantics,
% shorten this part
modeled as abstract trees. Complications occurring in
natural languages, such as agreement, case and word order, are abstracted away. 
That is, the abstract grammar contains no actual linguistic information, but
information about what categories are used and how to combine them into trees.
To add a new language it amounts to give a concrete grammar, that implements
the abstract grammar and describes how to translate abstract
trees into strings of the language. The framework enables us to parse strings into 
abstract trees as well as linearize trees into strings.
%Since all languages supported in an application share a common abstract syntax,
%we get multilinguality for free.

The grammar acts as an independent module and may be used in different
projects.
Reusability is further supported by the separation between resource grammars
and application grammars. The resource library provided with GF, 
implements morphological and
syntactical rules for more than 20 different languages.
Hence a writer of an application grammar can start her work at a higher
level and does not need to describe how to 
form standard sentences, phrases or decline words.

Since many of the languages in the GF library resemble each other grammatically,
they can share much of their grammar implementations. This is usually done by using a
\verb|Functor|, which lets a number of languages share parts of their implementation. 
In addition to simply avoiding code duplication, this technique aids the code maintenance.

GF has so far been used in a number of projects, MOLTO\cite{molto}, TALK\cite{talk}
and WebAlt\cite{webalt} to mention a few. 
All those are special domain applications, dealing with controlled natural
language.
This project takes a different approach by using GF for open domain language.
While the resources have to be general and language independent,
language specific constructs such as stylistic changes, idioms, informal
expressions are given in a special module.
This module, together with the Swedish resource grammar has been our starting
point. % of this project.
%By starting from an already existing Swedish grammar written in GF,
We get a basic description of the language and the framework provides
tools such as parsing, generation and
a well-tested interpretation of the parse trees. Furthermore, there are tools
for using the grammar in a number of programming languages like Haskell
and Java. 



\subsection{Talbanken}
Talbanken\cite{talbanken} is
a Swedish treebank assembled in the 1970s at Lund University.
In 2005 it was modernized by Nivre, Nilsson and Hall\cite{talbanken05} and
enriched with annotation for a full phrase structure analysis. 
We are using the treebank for development and evaluation. 

Although Talbanken contains both written and spoken Swedish,
only the prose material, consisting of 6316 sentences, will be used in this project.
This part was also used when training the data-driven parser Maltparser \cite{malt}. \\
%and the parsetree generated by our grammar will eventually be 
%compared to those from the treebank.

\subsection{Saldo}
SALDO\cite{saldo} is an open source lexicon resource
based on Svenskt Associationslexikon. It is
developed at Språkbanken at Gothenburg University
and intended for usage in language technology
research. 
From SALDO, a large GF lexicon has earlier been extracted,
containing 50 000 entries. By updating this technique, we want
to reimport SALDO, which has been developed since.
The importing method should be fast and reliable
enough to allow us to always have a fresh version of the dictionary
in GF.

%compare!!
\section{Related work}
Language technology for Swedish is an area of much interesting research.
There are tools for lexical extractions, such as
Extract for supervised lexicon extraction and
FM\cite{MarkusForsberg2007} for programming lexical resources.
There are two other deep grammar-based parser for Swedis,
SICS (Manny Raynen, ) and ? (Mats Wirén,?). Unlike our parser, 
they are not freely commercialized.
Among other parsers for Swedish, the statistical MaltParser\cite{malt}
which was trained on Talbanken, is worth mentioning. 
CassSwe (Kokkinakis and Johansson-Kokkinakis, 1999) is based on finite state cascades,
whereas the shallow parser GTA (Knutsson, Bigert and Kann, 2003) relies on rules of 
a context free grammar. Both GTA and CassSwe operate on POS tagged text.


\section{Work in progress}
\label{sec:progress}
\subsection{A tool for lexical acquisition}
To enlarge the lexicon, 
we have created a tool for automatic acquisition. We have
tested it on verbs with good results. It makes use of
the \emph{smart paradigm} given in the Swedish resource grammar.
The smart paradigm is a function that given one form of a word can
infer which paradigm it most likely belongs to.
For verbs, the paradigm accepts words in present tense indicative form.
If needed, it also accepts more verb forms showing the correct inflection.

By combining this method with the information from the tags in Talbanken,
the tool interactively generates GF lexicons. 
Given a list of words, it iteratively
tries to figure out how to conjugate each of them. If several forms of a word are 
given, the program will try to identify the one that carries the most linguistic
information, put this in a form recognized by the smart paradigm and ask GF to output
a table with the resulting inflection. 
If the table contains all other conjugations from the input list,
the program will ask the  user to
validate the claimed conjunction. The user may now either
allow the word to be added to the lexicon, remove it or request another guess.
Although using simple techniques, the tool 
manages to correctly guess 70-75\% of the given lemmas when tested on
verbs.\\
%Extract uses a similar 
%method but defines the paradigms itself.
%rules are given in another format.



%% Write more here!!
\subsection{Mapping of trees}
%R: mapping - very nice - more about it
The information from the tags in Talbanken can be used for many purposes.
We are currently working on an automatic transformation of Talbanken trees 
to trees in GF format. The translation makes use of the POS tags as well as
the syntactic information and the mapping has so far turned out to be unambiguous. 

The information given by this mapping, namely data about which form a word is
currently used in, may be used for the lexical extraction
tools.% Those can be enhanced if they are given more data. 
The translated trees will also enable us to extract probabilities for how often
different functions are used, a feature that would enable disambiguation.
Furthermore, the translation makes it easy to identify grammatical constructions
missing from the GF grammar and shows how the GF analysis differs from the one made
in Talbanken.

Another important use of the mapping is evaluation of the parser, which can be
accomplished by comparing the parse trees and the trees from the transformer.
%is an interesting way of
%% an straightforward
%accomplishing an evaluation.

% add evaluation

\subsection{Development of the grammar}
For Swedish, about 85\% of the GF resource code is shared with the other Scandinavian
languages. % >> When trying to achieve a lightweight, usable code? Not
%due to the resemblances between the languages, they share
%>> using a functor for sharing code, for when having the same structure but
%>> but only different words. 
However, if we aim for a deeper and more covering analysis of Swedish,
the implementation of the languages needs to be more independent.
%<<in order to avoid ad-hoc solutions and 
%Some constructions that are standard in
%Norwegian and Danish may be very rare in Swedish and while extending the
%grammar there may occur situations where a deeper restructuring of the
%implementation is needed. \\
A number of grammatical constructions have been added the the Swedish grammar
include constructs that are generally not present in other languages,
such as the reflexive pronoun \emph{sitt}: \\
\emph{Han såg \textbf{sitt} hus} $\; \; \; \;$ (\emph{He saw his (own) house}) $\;$
as opposed to \\
\emph{Han såg \textbf{hans} hus} $\;$ (\emph{He saw his (an other person's) house}) \\
%>> not critical, does not have to be in resource grammar
%>> also not expressable in all languages
The possibility to put a part of a sentence in focus has also been added:\\
\emph{\textbf{Glad} var han inte.} $\;$ (\emph{\textbf{Happy} was he not}). \\
Emphasizing words or phrases by putting them in the beginning of a sentence
is commonly used in Swedish.
This sort of rephrasing is however not given by the resource grammar, since it
has little effect on the logical representation.

In order to keep the grammar clean and to keep it syntactically correct,
much care is taken to develop it as neatly as possible. 

\section{Evaluation and future work}
%Compared to a statistical parser which operates like a black box,
%a rule-based one is not only theoretically interesting,
%but could also give a more explanatory output since the rules
%are given their names by a human and hence act as informative labels.
We are building a parser which is based on grammatical rules rather than
statistics. The parser will therefore validate that
a given sentence is grammatically correct according to the rules defined.
Chosen parts of the parser or grammar may also be used in other applications,
which deal with controlled natural languages. Automatic language
generation from the grammar is also provided by GF.

%% more explaination of probabilities
After having developed the grammar and lexicon, we plan to 
make the parser robust by using techniques such as chunk parsing, named
entity recognition and methods for handling unknown 
grammatical constructions such as idioms and ellipses. %etc. 
It is already possible to add probabilities for GF functions,
which rank the parse trees.
By adding dependency probabilities, we aim to improve the
disambiguation.

We intend to evaluate the parser both automatically - by comparing the output of
the translated trees from Talbanken - and manually by an expert in Swedish grammar.
Also the intermediate results are evaluated by our expert.


\section{Conclusion}
We intend to implement a robust deep parser for Swedish, by first developing
a large scale-grammar. The grammar can be compiled into a
parser by GF, and we will equip this with named entity recognition, a
voluminous lexicon, statistical information for disambiguation,
a method of parallel chunk parsing etc.
So far we have been working on the grammar, the lexical resources and
a translation of the treebank Talbanken to GF.
% adventage to what??
The usage of Grammatical Framework gives us the advantage to start from
a well-defined system of describing grammar, as well as tools for
parsing.


\section{Acknowledgments}
The work has been funded by Center of Language Technology.
We would also like to give special thanks to Aarne Ranta, % Ramona Enache,
Elisabet Engdahl, Krasimir Angelov, Olga Caprotti and Lars Borin for their help
and support.



\nocite{*}
\bibliographystyle{eptcs}
\bibliography{ourbib}
\end{document}


