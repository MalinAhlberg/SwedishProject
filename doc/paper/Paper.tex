\documentclass[submission]{eptcs} %rätt lincens?
\providecommand{\event}{LREC2012}
\usepackage[utf8]{inputenc}
%\usepackage{breakurl}        
\usepackage{graphicx}

\title{First results of developing a Swedish parser based on a wide coverage grammar} %A Wide-Coverage Swedish Grammar} % ska vi vara anonyma?
\author{Malin Ahlberg
% vad ska stå här eg?
\institute{Center of Language Technology, Gothenburg University, Sweden}
%\email{ahlberg.malin@gmail.com}
}
\def\titlerunning{First results of developing a grammar based Swedish parser}
\def\authorrunning{M. Ahlberg}
\begin{document}
\maketitle

\begin{abstract}
This paper describes an on-going work of a rule-based, open source
parser for Swedish. The central component will be a wide-coverage grammar
implemented in the grammar formalism Grammatical Framework (GF). The
resulting parser will be the first deep Swedish parser based on a grammar and
will give a full syntactic analysis of the input sentences.
In addition to GF, we use two other main resources; the Swedish
treebank Talbanken and the electronic lexicon SALDO.
% ha med detta?
The grammar will be highly reusable, chosen 
parts of the it can be used for applications dealing with controlled 
or free natural language. GF also provides libraries for compiling
grammars to a number of programming languages.
% allow easy interaction with other programs and grammars
% 
%>>
%>> something very commercial, interesting and outstanding here please!

%R: this phrase "We aim to implement a parser for open domain language
%and this paper describes our work and results so far as well as the
%directions of future work." doesn't belong in the abstract because it
%just describes the normal structure of any paper, without saying why
%your work are interesting/outstanding, worth-reading-further. Make it
%sound more commercial, if you want to have it there.


\end{abstract}

\section{Introduction}
%R: the first sentence - in "deep robust parser" - why "deep" - grammar
%formalism, why "robust" - ??
% afterwards, you don't actually have named-entity recognition, and
%further more, there's no such thing as "exhaustive lexicon" - you can
%always switch to another domain and find new words. Further on, the
%work has nothing to do with statistitics so far, so you can't mention
%as opening line.  The same goes for parallel chunk parsing.

%R: the first sentence doesn't really make sense because you first
%refer to a parser and only then to GF and a grammar, but without
%giving an explanation of how making the grammar would give you a
%parser for free.

Swedish is a North Germanic language spoken by approximately 10 million people.
It is a SVO language, and although the word order is relatively strict,
inverted order is commonly used
to indicate questions or to emphazise different parts of the sentence. \\
Language technology for Swedish is an area of much interesting research.
%on  are currently being conducted. \\
%R: -- the second paragraph - OK but please extend.
Our goal is to implement a wide coverage grammar and parser for Swedish.
We thereby investigate how the grammar formalism Grammatical Framework
%further described in section \ref{sec:gf}, 
can be used for open domain parsing.
By starting from an already existing Swedish grammar written in GF, we get 
an fundamental description of the language. The framework also provides
tools such as parsing, generation and
a well tested interpretation of the parse trees. Furthermore, there are tools
for using the grammar in a number of programming languages. 
%>> , tools for visualizing ...

From the GF grammar we aim to implement a robust parser. 
The parser will be evaluated both by experts and by comparing the results of
parsing sentences in an often used Swedish treebank, Talbanken.

This paper will briefly introduce GF, Talbanken and SALDO in section \ref{sec:background}
and describe our current work in section \ref{sec:progress}.
%We aim to implement % The goal is to be able to 
%a parser for open domain language and 
%This paper describes our work and results 
%so far as well as the directions of future work.

\section{Background}
\label{sec:background}
\subsection{Grammatical Framework}
\label{sec:gf}

%R: the GF part - "especially suitable" - it's actually meant for the purpose
%:)

%compiler inspired approach, 
% om resources: library that implements morphology and syntax rules, länk till papper om dessa
% skriv ord som modular, architectur, reusable (fast stava dem rätt)

Grammatical Framework\cite{ranta-2011} (GF) is a grammar formalism based on functional
programming, designed for multilingual grammar applications.

The key idea
is to divide the grammar into abstract and concrete parts. %% parts?
The abstract grammar gives a logical representation of the semantics,
%of the domain, 
modeled as abstract trees. Complications occurring in
natural languages, such as agreement, case and word order, are abstracted away. 
%R: what complications do you abstract away ?
%categories and trees are not interesting as such, but their usage
%is - why do you need them in the abstract syntax actually ?
That is, the abstract grammar contains no actual linguistic information, but
information about what categories are used and how to combine them into trees.

To add a new languages means to give a concrete grammar. This implements
the abstract grammar and describes how to translate abstract
trees to strings of the language. The framework enables us to parse strings into 
abstract trees as well as linearize trees into strings.
Since all languages supported in an application shares a common abstract syntax,
we get multilinguality for free.

The grammar acts as an independent module and may be used in different
projects.
%R: more about the GF library - if you have space
Reusability is further supported by the separation between resource grammars
and appliciation grammars. The resource grammars is 
a library provided with the GF package, 
where information about the morphological and
syntactical rules for more than 20 different languages are implemented.
Hence a writer of a application grammar can start her work at a higher
level and does not need to describe how to 
form standard sentences, phrases or decline words.
%, but can focus on the
%particularities for the given domain and choose what kind of 
%utterances that should be allowed.

%so far bara used för vissa saker, 
% about why we need more, extra module, separate swedish from scandinavian
% get inspired by Shafqat.
GF has so far been used in a number of projects, MOLTO\cite{molto}, TALK\cite{talk}
and WebAlt\cite{webalt} to mention a few. 
All those are special domain applications, dealing with controlled natural
language.
This project experiments on other usage of a GF grammar - to use it for open
domain language.

% parsing free language, more work is required and it has been investigated what
%Further, other tools that are needed have been implemented and examined.
%will be needed to make it sufficient and a start on developing tools has beenmade. 


\subsection{Talbanken}
%Wherefrom, how big, who, when, version, used for maltparser
Talbanken\cite{talbanken} is
a Swedish treebank put together in the 1970s at Lund University.
In 2005 it was modernized by Nivre, Nilsson and Hall\cite{talbanken05} and
enriched with annotation for a full phrase structure analysis. 
We are using the treebank for development and evaluation. 

Although Talbanken contains both written and spoken Swedish,
%For this project 
only the prose material, consisting of 6316 sentences will be used in this project.
This part was also used when training the data-driven parser Maltparser \cite{malt}. \\
%and the parsetree generated by our grammar will eventually be 
%compared to those from the treebank.

\subsection{Saldo}
%R: Saldo part - why would you like to reimport ? "enlarge and enhance"
%- how do you enhance it ?
SALDO\cite{saldo} is an open source lexicon resource
based on Svenskt Associationslexikon (SAL). It is
developed at Språkbanken at Gothenburg University
and intended for usage in language technology
research. 
% how it is tagged, other usages, benefits,
% how easy it is to map it to gf
From SALDO, a large GF lexicon
has earlier been extracted,
containing 50 000 entries.
By updating this technique, we would like
to reimport SALDO, which has since then been developed.
%and we would like
%to reimport the it using the same
%updated techniques. %in order to enlarge our lexicon,
%we  in order to enlarge and
%GF version. 
The importing method should be fast and reliable
enough to be allow us to always have a fresh version of the dictionary
in GF.


\section{Work in progress}
\label{sec:progress}
\subsection{A tool for lexical acquisition}
%R: lexicon extraction part - more about everything, comparison with
%Extract and Functional Morphology, evaluation of the lexicon builder -
%"supervised lexicon extraction"(common term for this - google it if
%you want to find related work)

To enlarge the lexicon, 
a tool for automatically acquisition has been created. It has 
been tested on verbs with good results. It makes use of
the \emph{smart paradigm} given in the Swedish resource grammar.
The smart paradigm acts as functions that given one form of a word can
infer which paradigm it most likely belong to.
The smart paradigm for verbs accepts words in present tense indicative form.
If needed, it also accepts more verb forms showing the correct inflection.
%which Making use of the

By combining this method with the information from the tags in Talbanken,
%and the paradigms in the Swedish resource grammar, 
the tool interactively generates GF lexicons. 
Given a list of words, it iteratively
tries to figure out how to conjugate each of them. If several forms of a word is 
given, the program will try to identify the one that carries the most linguistic
information, put this in a form recognized by the smart paradigm and ask GF to output
a table with the resulting inflection table. 
If this table contains all other forms from the input list,
the program will ask the  user to
validate the claimed conjunction. The user may now either
allow the word to be added to the lexicon, remove it or demand the program
to make another guess.
Although using simple techniques, the tool 
manage to correctly guess 70-75\% of the given lemmas when tested on
verbs. \\
For further extractions, there are also earlier developed tools such as
Extract\cite{extract} for supervised lexicon extraction and
FM\cite{fm} for programming lexical resources.
%Extract uses a similar 
%method but defines the paradigms itself.
%rules are given in another format.



%% Write more here!!
\subsection{Mapping of trees}
%R: mapping - very nice - more about it
The information from the tags in Talbanken can be used for many purposes.
We are currently working on an automatic transformation of Talbanken trees 
to trees in GF format. The translation makes use of the POS tags as well as
the syntactic information and the mapping has so far turned out to be unambiguous. 

The information given by this mapping may be used for the lexical extraction
tools. Those can be enhanced if they are given more data about which form a word is
currently used in. 
The translated trees will also enable us to extract possibilities for how often
different functions are used, a feature that would enable disambiguation.
Furthermore, the translation makes it easy to identify grammatical constructions
missing from the GF grammar and shows how the GF analysis differs from the one made
in Talbanken.

%>> by looking on how often a gf function is used
%<< that would improve the parsing considerably.
Another important use of the mapping is evaluation. By comparing the trees
from the parser and from the transformer, we get an interesting way of
accomplishing an evaluation.
%An evaluation of the parser
%could also be accomplished by  \\
%<< brastartord there are differences in the notation of the sättet träden är uppbyggda på \\
%>>maybe something about how it's going and why it's hard?
%>> exempel?? \\ 

%explation of the trees
%why we need the mapping : evaluation, adding words, finding missing constructions
%difficulties

\subsection{Development of the grammar}
%R: New things and hard things in the grammar" - this goes hand in
%hand with more explanations about resource library and functors for
%families, also emphasize the words that you refer to in the example -
%bold font or smth equally noticeable. Explain more why these
%constructions weren't there before.

For multilingual GF applications, translation should always be possible, and
therefore the resource abstract may only contain constructions that
are common to all implemented languages.
%>> resources for controlled language. does not aim for covering whole language
%>> googla efter bra citat
%This means that everything added to the resources must be expressable in all other languages.
Therefore, the resources have to be general and non language specific.
Language specific constructs such as stylistic changes, idioms, informal
expressions etc may be given in a special module.
This module, together with the Swedish resource grammar has been the starting
point of this project.

Since many of the languages in the GF library reassemble each other grammatically,
they can share much of their grammar implementations. This is usually done by using a
\verb|Functor|, which lets a number of languages share parts of their implementation. 
In addition to simply avoiding code duplication, this technique aids the code maintenance.
For Swedish, about 85\% of the resource code is shared with the other Scandinavian
languages. % >> When trying to achieve a lightweight, usable code? Not
%due to the resemblances between the languages, they share
%>> using a functor for sharing code, for when having the same structure but
%>> but only different words. 
However, if we aim for a deeper and more covering analysis of Swedish,
%For a bigger and still well-defined Swedish grammar however,
the implementation of the languages needs to be more separated. 
% in order to avoid ad-hock solutions.
%<<in order to avoid ad-hoc solutions and 
%Some constructions that are standard in
%Norwegian and Danish may be very rare in Swedish and while extending the
%grammar there may occur situations where a deeper restructuring of the
%implementation is needed. \\
A number of grammatical constructions have been added the the Swedish grammar.
%>> this is language specific, therefore not in common abstract
Those include constructs that are generally not present in other languages,
such as the reflexive pronoun \emph{sitt}: \\
\emph{Han såg \textbf{sitt} hus} $\; \; \; \;$ (\emph{He saw his (own) house}) $\;$
as opposed to \\
\emph{Han såg \textbf{hans} hus} $\;$ (\emph{He saw his (an other person's) house}) \\
%>> an stylistic change that has been added, not needed for expressable new things (typ) 
%>> not critical, does not have to be in resource grammar
%>> also not expressable in all languages
The possibility to put a part of a sentence in focus has also been added:\\
\emph{Glad var han inte.} $\;$ (\emph{Happy was he not}). \\
This sort of rephrasing which has little effect on the logical representation
is commonly not given by the resource grammar.

In order to keep the grammar as clean and to keep it from allowing syntactical errors,
much care is taken to develop this as neatly as possible. 
%<<Some ellipses are added, while there is work going on to add trickier ones (eller inte 
%<<ellipser just nu dårå, men vi skriver något fint här med, ja-jaaaa!).

\section{Related work}
There are other parser for Swedish parser such as the the statistical MaltParser\cite{malt}
which was trained on Talbanken. 
CassSwe (Kokkinakis and Johansson-Kokkinakis, 1999) is a  based on finite state cascades,
whereas shallow parser GTA (Knutsson, Bigert and Kann, 2003) relies on rules of 
a context free grammar. Both GTA and CassSwe operates on POS tagged text.
%deep Swedish Core Language Parser,
%shallow Swedish Constraint Grammar. 


%% Rephrase
\section{Evaluation}
%R: you need especially an evaluation part where you can also argue
%that the method is better than statistics.
Compared to a statistical parser which operates like a black box,
a rule-based one is not only theoretically interesting,
but could also give a more explanatory output since the rules
are given their names by a human and hence acts as informative labels.
Chosen parts of the parser or grammar may also be used in other applications,
which may be dealing with controlled natural languages. Automatic language
generation from the grammar is also provided by GF.

We intend to evaluate the parser both automatically - by comparing the output
the translated trees from Talbanken - and manually by an expert in Swedish grammar.

%% more explaination of probabilities
\section{Future Work}

After having developed the grammar and lexicon, we plan to 
make the parser robust by using techniques such as chunk parsing, named
entity recognition, methods for handling unknown 
grammatical constructions such as idioms and ellipses. %etc. 

It is already possible to add probabilities for GF functions,
which serves as a ranking when several parse trees are possible.
By adding dependency probabilities, we would like to improve the
disambiguation.


% ha med det som inte nämnts innan
% ha en jämförelse med stat.parse?
% är den deep?
\section{Conclusion}
We intend to implement a robust deep parser for Swedish, by first developing
a large scale grammar and then equip this with named entity recognition, a
voluminous lexicon, statistical information for disambiguation,
a method of parallel chunk parsing etc.
So far we have been working on the grammar, the lexical resources and
a translation of the treebank Talbanken to GF.
The usage of Grammatical Framework gives us the advent age to start from
a well-defined system of describing grammar, as well as tools for
using the grammar in combination with programming languages like Haskell
and Java. 


\section{Acknowledgments}
The work has been funded by Center of Language Technology.
I would also like to give special thanks to Aarne Ranta, Ramona Enache,
Elisabet Engdahl, Krasimir Angelov and Lars Borin for their help and support.



\nocite{*}
\bibliographystyle{eptcs}
\bibliography{ourbib}
\end{document}


