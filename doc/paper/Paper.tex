\documentclass[submission]{eptcs} %rätt lincens?
\providecommand{\event}{LREC2012}
\usepackage[utf8]{inputenc}
\usepackage{graphicx}

\title{%First results of developing a Swedish parser based on a wide-coverage grammar\\
        Combining language resources into a grammar-driven Swedish parser} 
        %or Bootstrapping a Swedish parser \\
        %or A grammar-based Swedish parser from already existing resources}
\author{Malin Ahlberg \qquad\qquad Ramona Enache
%\and
%Co Author \qquad\qquad 
\institute{Department of Computer Science \& Engineering, Gothenburg University, Sweden}
}
\def\titlerunning{Combining language resources into a grammar-driven Swedish parser}
\def\authorrunning{M. Ahlberg}
\begin{document}
\maketitle

\begin{abstract}
This paper describes on-going work on a rule-based, open-source
parser for Swedish. The central component is a wide-coverage grammar
implemented in the GF formalism (Grammatical Framework), from which
we obtain a %can be compiled to a % from which we derive?? 
parser.
In addition to GF, we use two other main resources: the Swedish
treebank Talbanken and the electronic SALDO lexicon.
The resulting parser 
gives a full syntactic analysis of the input sentences.
It will be highly reusable, freely available,
and as GF provides libraries for compiling
grammars to a number of programming languages,
chosen parts of the the 
grammar may be used in various NLP applications. % dealing with
%controlled or free
%natural language. 
% allow easy interaction with other programs and grammars
%>> something very commercial, interesting and outstanding here please!

\end{abstract}

\section{Introduction}
Swedish is a North-Germanic language spoken by approximately 10 million people.
It is a verb-second language: the second constituent of a declarative main
clause must consist of a verb.
The first constituent of the clause is usually made up of the subject,
although it likewise could consist of adverbial phrases or objects.
In questions and subordinate clauses, inverted word order occurs.

Our goal is to implement a wide-coverage grammar and parser for Swedish
using the GF formalism, and 
thereby investigate how GF can be used for open-domain parsing.
% better 
We compile a large-scale grammar to a parser
and combine it with the extensive lexicon SALDO. The Swedish treebank 
Talbanken provides manually tagged trees which we use for improving and evaluating
the grammar. The parser will additionally
be evaluated by experts. \\
%s and by comparing the results of parsing sentences in a commonly used Swedish treebank, Talbanken.
We chose our resources so that both our grammar and parser can 
be freely available and open-source. 

This paper will briefly introduce GF, Talbanken and SALDO in Section \ref{sec:background}
and describe our results in Section \ref{sec:progress}.
%We aim to implement % The goal is to be able to 
%a parser for open domain language and 
%This paper describes our work and results 
%so far as well as the directions of future work.

\section{Background}
\label{sec:background}
\subsection{Grammatical Framework}
\label{sec:gf}

%städa upp detta stycke!!

Grammatical Framework\cite{ranta-2011} (GF) is a grammar formalism based on functional
programming. %, designed for multilingual applications.

The key idea is to divide a grammar into abstract and concrete parts. %% parts?
The abstract grammar gives a logical representation of the semantics,
% shorten this part
modeled as abstract trees.
%Complications occurring in natural languages, such as agreement, case and word
%order, are abstracted away. 
%That is, 
%It contains no actual linguistic information, but
%information about what categories are used and how to combine them into trees.
The concrete grammars tell how to translate the abstract trees to a given language,
and deal with issues such as word order, case and agreement. 
%implements the abstract 
%To add a new language it amounts to give a concrete grammar, that implements
%the abstract grammar and describes how to translate abstract
%trees into strings of the language.
The framework enables us to parse strings into 
abstract trees as well as linearize trees into strings.
%Since all languages supported in an application share a common abstract syntax,
%we get multilinguality for free.

The grammar acts as an independent module and
%and may be used in different projects.
reusability is further supported by the separation between resource grammars
and application grammars. The resource library provided with GF 
implements morphological and
syntactical rules for more than 20 different languages.
Hence the writer of an application grammar can start her work at a higher
level and does not need to describe how to 
form standard sentences, phrases or inflect words.

Since many of the languages in the GF library resemble each other grammatically,
they can share much of their implementations. This is usually done by using a
\verb|Functor|, % , which lets a number of languages share parts of their implementation. 
avoids code duplication and aids the code maintenance.

GF has so far been used in a number of projects, MOLTO\cite{molto}, TALK\cite{talk}
and WebAlt\cite{webalt} to mention a few. 
All those are special domain applications, dealing with controlled natural
language.
This project takes a different approach by using GF for open domain language.
%While the resources have to be general and language independent,
%language specific constructs such as stylistic changes, idioms, informal
%expressions are given in a special module.
%This module, together with 
Using the Swedish resource grammar as our starting
point, % of this project.
%By starting from an already existing Swedish grammar written in GF,
we get a basic description of the language. The framework provides
tools such as parsing, generation and
a well-tested interpretation of the parse trees. Furthermore, there are tools
for using GF grammars in a number of programming languages like Haskell
and Java. 



\subsection{Talbanken}
For development and evaluation, we use the Swedish treebank
Talbanken\cite{talbanken}.
It was assembled in the 1970s at Lund University and modernized
in 2005 by Nivre, Nilsson and Hall\cite{talbanken05} and
enriched with annotation for a full phrase structure analysis.  \\
Although Talbanken contains both written and spoken Swedish,
only the prose material, consisting of 6316 sentences, will be used in this project.
This part was also used when training the data-driven parser Maltparser \cite{malt}. \\
%and the parsetree generated by our grammar will eventually be 
%compared to those from the treebank.

\subsection{Saldo}
SALDO\cite{saldo} is an open source lexicon resource
based on Svenskt Associationslexikon. It is
developed at Språkbanken at Gothenburg University
and intended for usage in language technology
research. 
We are developing tools for extracting GF lexicons
from SALDO, which have resulted in a dictionary containing
over 98000 entries including information about verbs that
are reflexive or needs a particle.
The importing method should be fast and reliable
enough to allow us to always have a fresh version of the dictionary
in GF.

%compare!!
\section{Related work}
Language technology for Swedish is an area of much interesting research.
There are two other deep grammar-based parser for Swedish:
Swedish Core Language Engine (Raynen and Gambäck, 1992) and a
unification-based chart parser\cite{wiren},
%correct reference?
though none of them are freely available.
Among other parsers for Swedish, the statistical MaltParser\cite{malt},
trained on Talbanken, is worth mentioning. 
CassSwe (Kokkinakis and Johansson-Kokkinakis, 1999) is based on finite state cascades,
whereas the shallow parser GTA (Knutsson, Bigert and Kann, 2003) relies on rules of 
a context free grammar. Both GTA and CassSwe operate on POS tagged text.\\
Extract and FM\cite{MarkusForsberg2007} are tools for supervised lexicon
extraction compatible with GF.
%For lexical extractions, there are tools like
%Extract for supervised lexicon extraction and
%FM for programming lexical resources.


\section{Results and progress}
\label{sec:progress}
\subsection{A tool for lexical acquisition}
To enlarge the lexicon, 
we have created a tool for automatic acquisition. We have
tested it on verbs, with good results. It makes use of
the \emph{smart paradigm} given in the Swedish resource grammar.
The smart paradigm is a function which given one form of a word, can
infer which paradigm it most likely belongs to.
For verbs, the paradigm accepts words in present tense indicative form.
If needed, it also accepts more verb forms showing the correct inflection.

By combining this method with the information from the tags in Talbanken,
the tool interactively generates GF lexical entries. 
Given a list of words, it iteratively
tries to figure out how to conjugate each of them. If several forms of a word are 
given, the program will try to identify the one that carries the most linguistic
information, put this in a form recognized by the smart paradigm and ask GF to output
a table with the resulting inflection. 
If the table contains all other conjugations from the input list,
the program will ask the  user to
validate the claimed paradigm. The user may now either
allow the word to be added to the lexicon, remove it or request another guess.

The tool has been tested on verbs. Although using simple techniques it 
manages to assign the correct paradigm to 70-75\% of the given lemmas.
A smaller test has shown that out of the accepted lemmas, the correct guess is
made directly in 75\% of the cases, whereas the user has to reject one or more
guesses for 25\%. 
%is % estimated
% whereas the user has to reject the others to force a new guess. %rest are found after an incorrect guess rejected by the user.\\
%Extract uses a similar 
%method but defines the paradigms itself.
%rules are given in another format.



%% Write more here!!
\subsection{Mapping of trees}
%R: mapping - very nice - more about it
The information from the tags in Talbanken can be used for many purposes.
We are currently working on an automatic transformation of Talbanken trees 
to trees in GF format. The translation makes use of the POS tags as well as
the syntactic information, and the mapping has so far turned out to be unambiguous. 

The mapping gives information about which form a word is
currently used in and this may be used by the lexical extraction
tools. % Those can be enhanced if they are given more data. 
The translated trees enable us to extract probabilities for how often
different functions are used, a feature that will enable disambiguation.
Furthermore, the translation makes it easy to identify grammatical constructions
missing from the GF grammar and shows how the GF analysis differs from the one made
in Talbanken.
Another important use of the mapping is evaluation of the parser, which can be
accomplished by comparing the parse trees and the trees from the transformer.

We have so far concentrated on shorter sentences, without idioms and
conjunction. 
Since GF demands information 
that is not given by the Talbanken tags -- such as the valency of verbs -- a
full translation requires that all words used are already known to GF. 
The evaluation of trees of this sort shows that the mapping
can restore 85\% of the nodes. 
%, given that all words are
%known to the grammar.
If we lift the restrictions, we can restore 63\% of the tags.
%mapping, a number that would probably increase substantially when working with a bigger lexicon.
%is an interesting way of
%% an straightforward
%accomplishing an evaluation.

% add evaluation

\subsection{Development of the grammar}
For Swedish, about 85\% of the GF resource code is shared with the other Scandinavian
languages. % >> When trying to achieve a lightweight, usable code? Not
%due to the resemblances between the languages, they share
%>> using a functor for sharing code, for when having the same structure but
%>> but only different words. 
However, if we aim for a deeper and more comprehensive analysis of Swedish,
the implementation of the languages needs to be more independent.
%<<in order to avoid ad-hoc solutions and 
%Some constructions that are standard in
%Norwegian and Danish may be very rare in Swedish and while extending the
%grammar there may occur situations where a deeper restructuring of the
%implementation is needed. \\
The resource grammar gives a good start and our present grammar 
covers constructions such as declarative sentences with normal or inverted
word order, questions, passives, imperatives, relative clauses, cleft
constructions etc.
A number of constructs that are generally not present in other languages and
therefore not given by the resources, have also been added. These include the use of
the reflexive pronoun \emph{sitt}: \\
\emph{Han såg \textbf{sitt} hus} $\; \; \; \;$ (\emph{He saw his (own) house}) $\;$
as opposed to \\
\emph{Han såg \textbf{hans} hus} $\;$ (\emph{He saw his (another person's) house}) \\
%>> not critical, does not have to be in resource grammar
%>> also not expressible in all languages
Fronting words or phrases is very common in Swedish and are now allowed by the grammar:\\
%the possibility for this has been added to the grammar:\\
\emph{\textbf{Glad} var han inte.} $\;$ (\emph{\textbf{Happy} was he not}). \\
This sort of rephrasing is not given by the resource grammar, since it
has little effect on the logical representation.

%In order to keep the grammar clean and to keep it syntactically correct,
%much care is taken to develop it as neatly as possible. 

\section{Evaluation and future work}
%Compared to a statistical parser which operates like a black box,
%a rule-based one is not only theoretically interesting,
%but could also give a more explanatory output since the rules
%are given their names by a human and hence act as informative labels.
We are building a parser which is based on grammatical rules rather than
statistics. The parser will therefore validate that
a given sentence is grammatically correct according to the rules defined.
Chosen parts of the parser or grammar may be used in other applications
which deal with controlled natural languages. Automatic language
generation from the grammar is also provided by GF.

%% more explanation of probabilities
We aim to make the parser robust by equipping it with techniques such as chunk
parsing, named entity recognition and methods for handling unknown 
grammatical constructions such as idioms and ellipses. %etc. 
It is already possible to add probabilities for GF functions,
which rank the parse trees.
By adding dependency probabilities, we aim to improve the
potential for disambiguation.

We intend to evaluate the parser both automatically -- by comparing the output of
the translated trees from Talbanken -- and manually by professor Elisabet Engdahl.
%an expert in Swedish grammar.
She also evaluates the intermediate results.


\section{Conclusion}
%We intend to implement a robust deep parser for Swedish, by first developing
%a large scale-grammar. 
By combining GF with a large lexicon, we acquire a deep parser for 
Swedish. All parts, including the underlying grammar, are open-source and can be re-used.
We have developed tools for extending the lexicon with words from Talbanken,
and we show how to make use of the information in the manually tagged
treebank.
The usage of GF allows us to start from
a well-defined system for describing grammar, as well as tools for
parsing.

%We have shown how to combine existing resources, we build a large-scale grammar for Swedish.
%The grammar can be compiled into a
%parser by GF, and we equip this with a extensive lexicon. 
%and we will equip this with named entity recognition, a
%voluminous lexicon, statistical information for disambiguation,
%a method of parallel chunk parsing etc.
%So far we have been working on the grammar, the lexical resources and
%a translation of the treebank Talbanken to GF.
% adventage to what??


\section{Acknowledgments}
The work has been funded by Center of Language Technology.
We would also like to give special thanks to Aarne Ranta, % Ramona Enache,
Elisabet Engdahl, Krasimir Angelov, Olga Caprotti, Lars Borin and John Camilleri for their help
and support.



\nocite{*}
\bibliographystyle{eptcs}
\bibliography{ourbib}
\end{document}


