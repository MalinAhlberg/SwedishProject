\documentclass[submission,copyright,creativecommons]{eptcs} %rätt lincens?
\providecommand{\event}{LREC2012}
%\usepackage{breakurl}        
\usepackage{graphicx}

\title{Den stora grammatiken} % ska vi vara anonyma?
\author{Malin Ahlberg
\institute{Gothenburg University, Sweden}
\email{ahlberg.malin@gmail.com}
}
\def\titlerunning{First results}
\def\authorrunning{M. Ahlberg}
\begin{document}
\maketitle


\begin{abstract}
om parsning of free language (take from old report)
but for linguistics, make a wide coverage grammar, robust parser
By using existing technologies
and tools:use three resources, gf, talbanken, saldo
A parser like this would be of great use for many natural
language processing applications, such as translation and information retrieval
as well as making semantic representations.
The goal is to be able to parse open domain language and the
parser will be evaluated on an extensive Swedish
treebank, Talbanken. 

This report describes the start of implementing a robust parser for Swedish,
using the grammar formalism Grammatical Framework\cite{gf} (GF).

\end{abstract}

\section{Introduction}

Making computers able of handling human language is a 
hard problem.
The meaning of a sentence depends not only of which words it consists of, but
also on their syntactic use, how they interact and relate to each other.
For a computer to make sense of natural language, it needs to analyse this 
syntactic structure; it needs a good grammar and parser.
Moreover, the parser needs a preprocessor, named entity recognizer and more
tequnices to make it robust. 

We are working on an extended implementation of a grammar from Swedish, allowing
complicated constructions such as focusing different parts of the sentences,
using complex relative clauses etc and constructions (byt ord) specific to Swedish.

The libraries of GF provide a basic grammar for
Swedish, covering the fundamental features of the language, such as morphology and
commonly used syntax. This is suitable 
for building domain specific applications. In those cases the user is not
allowed to freely compose sentences, but has to stay within the bounds of a 
controlled language. Both the vocabulary and grammatical structures are fixed,
meaning that there only is a limit number of ways to write a sentence.
Parsing open domain natural language is a much bigger task, since it involves handling both
standard and non-standard grammatical constructions. 

The future work will include extending and enhancing the existing Swedish GF grammar,
importing lexicon and develop techniques for handling unknown words and grammatical
constructions, proper names, idioms, ellipses etc, in order to make the parser robust.

By using this, we hope to eventually implement a grammar able to handle free % synonyyym!
Swedish. It would be of great use for many natural
language processing applications, such as translation and information retrieval
as well as making semantic representations.


\subsection{Grammatical Framework}

%compiler inspired approach, 
% om resources: library that implements morphology and syntax rules, länk till papper om dessa
% skriv ord som modular, architectur, reusable (fast stava dem rätt)

% about why we need more, extra module, separate swedish from scandinavian
GF\cite{gf} (Grammatical Framework) is a grammar formalism based on functional programming,
and suitable for multilingual grammar applications. The key idea is to have
one abstract grammar, modelling the structure of the domain. 
In addition, concrete grammars are implementing the abstract in different languages.
Since all languages supported in an application shares a common abstract syntax,
multilingual is easy to support.

In this way the grammar becomes independent, it may be reused in another project
and the implementation may be changed without modifying the application utilizing the grammar.

GF provides resource grammars\cite{resource} implementing information about the morphological and
syntactical rules for 18 different languages. By utilizing these, 
GF is an advantageous tool for writing special-purpose grammars.

  
The libraries of GF, the resource libraries\cite{resource}, provides
a common abstract and implementations of this in 18 languages.
Since translation always should be possible, the abstract may only contain constructions that
are common to all implemented languages.

  But this also means that all ---- added to the resources
  must be translatable to and expressable in all other language.
 therefor, the resources are very general, non language specific.

Language specific constructs may be
given in the module \verb|Extra.gf|. This may be stylistic changes, idioms, informal
expressions etc.
As an example, the Swedish \verb|Extra| module contains a function for expressing sentences
where the negation is put in focus: \emph{``Inte var jag glad"} (\emph{``Not was I happy"}).


The work to enhance and expand the Swedish GF grammar has already been started
% parsing free language, more work is required and it has been investigated what
%Further, other tools that are needed have been implemented and examined.
%will be needed to make it sufficient and a start on developing tools has beenmade. 


\section{Talbanken}
%Wherefrom, how big, who, when, version, used for maltparser
Talbanken\cite{talbanken} is
a Swedish treebank which was put together in the 1970s at Lund University,
consisting of 6316 sentences. Each word is tagged with 
its part of speech, syntactic function and its position in a head dependent
analysis.
In 2005 Talbanken was mordenized \cite{} and enriched with annotation for a
full phrase structure analysis. Talbanken05 was used when training the 
data-driven parser Maltparser \cite{}, which var bra?

\section{Saldo}

SALDO\cite{saldo} is an open source lexicon resource
based on Svenskt Associations Lexikon (SAL). It is
developed at Sprakbanken\cite{} at Gothenburg University
and intended to be used in language technology
research. 
% how it is tagged, other usages, benefits,
% how easy it is to map it to gf
From SALDO, a large GF lexicon, \\
\verb|DictSwe.gf| has earlier been extracted \cite{},
containing 50 000 entries. Using the same updated tecniques,
we would like to reimport the lexicon in order to enlarge,
enhance it. The importing-method should be fast and reliable
enough to be able to always have an fresh version of the dictonary
in GF.
We are using parts of Talbanken for testing and development,
and the parsetree generated by our grammar will eventually be 
compared to those from the treebank.


\section{The lexicon tool}
A tool for automatically acquisition has been created. Making use of the
tags in Talbanken and the paradigms in the Swedish resource grammar, 
it interactivelly generates GF lexicons. The user
verifies tho correctness of a guess from the program, and can either
allow the word to be added to the lexicon, remove it or demand the program
to make another guess.
Altough using very simple tecniques, the test of the tool on verbs 
has a correctness rate of about
70-75\%, and can easily ? be improved.

use together with Extract/FM,

\section{Mapping of trees}
The information from the tags in Talbanken can be used for many xx.
The tools for lexical extraction can be enhanced if we get more information
about which form the word is currently used in. 
We are currently working on a automatical mapping from Talbanken trees to trees in GF format.
This would later enable us to extract possibilities for how often different functions
are used, a feature (!) that would improve the parsing considerably.
The evalution of the parser could also be accomplished by comparing the trees
from the parser and from the mapping-process.
brastartord there are differences in the notation of the sättet träden är uppbyggda på \\
 exempel?? \\ 

explation of the trees
why we need the mapping : evaluation, adding words, finding missing constructions
difficulties

\section{New things and hard things in the grammar}
Due to the ressemblemences between the Scandinavian languages, they share about 80\% (?)
of their resource code. This is good because blabla.
However, for a bigger Swedish grammar, the implementation of the languages need to be more separated,
in order to avoid ad-hoc solutions and constructions that are standard in
Norweigan and Danish but very rare in Swedish. \\ %skriv om här änna
A number of grammatical constructions have been added the the Swedish grammar.
Those include using the reflexive pronoun \emph{sitt}:\\
\textbf{Han såg sitt hus} \emph{He saw his (own) house} \\
as opposed to \\
\textbf{Han såg hans hus} \emph{He saw his (an other person's) house} \\
% to simple for linguists?
The possibility to put a part of a sentence in focus has also been added such as:\\
\textbf{Glad var han inte} \emph{Happy was he not}
Some ellipses are added, while there is work going on to add trickier ones (eller inte 
ellipser just nu dårå, men vi skriver något fint här med, ja-jaaaa!).


\section{Future Work}
robust parser, ner, preprocessing, adding new words,
probabilties
ellipses, chunk parsing 


\section{Conclusion}
hard but nice
why better than statistical


\nocite{*}
\bibliographystyle{eptcs}
\bibliography{ourBib}
\end{document}


