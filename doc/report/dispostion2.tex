\documentclass{article} 
\usepackage[utf8]{inputenc}
\usepackage{graphicx}

\begin{document}
% ska gå att förstå och locka till läsning
%ska tydligt beskriva vad rapporten handlar om
%ska innehålla vettiga sökord
\title{Dispositon for master thesis \\
             \small{A large scale grammar and parser for Swedish}}
\author{Malin Ahlberg}
\maketitle

\section{Introduction} 
Short introduction to parsing, GF, Swedish, controlled natural language vs.
free language.
Why we want a large scale grammar and a parser, what we can get from GF and
what we can't get. Which parts we are focusing on and why.

\section{Background}  
Longer introduction to GF, Talbanken and Saldo. Other parsers and related work.

\section{Progress}  
\begin{itemize}
\item{
One part about how we have imported saldo, problems and results, how we deal
with words that were not imported. Use of lexical aquisition tools. 
}
\item{
One part about the mapping from Talbanken to GF, description. About the purpose.
This part explains
the tags in Talbanken and gives examples of how they are translated,
examples of what can be parsed and what cannot.
}
\item{
One part about the grammar. The state of the original grammar, what was covered and not. 
What parts that I have been working on and why, 
what constructions I wanted to add, examples of sentences that should work
and of sentences that shouldn't be covered. How to separate Swedish from
Scandinavian, bigger reconstructions in the grammar. 
}
\end{itemize}

\section{Results and discussion}
Evaluation methods and results from all parts. 

Discussion of the results and methods, why are the results good, why are they not better.
What other methods could have been used? What did we/I expect and what happend?


\section{Future Work}
What is still left to do? Description and direction of future work:
how to make the parser robust, what is still needed in the grammar, 
how to get valency information, make the parser faster.
Getting and using probabilities.


\section{Conclusion}
My conclusions about the work.
\end{document}
