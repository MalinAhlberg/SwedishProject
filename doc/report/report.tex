\documentclass{report} 
\usepackage[utf8]{inputenc}
\usepackage{graphicx}
\usepackage[margin=1.5in]{geometry}
\usepackage{lingmacros}


\begin{document}
% ska gå att förstå och locka till läsning
%ska tydligt beskriva vad rapporten handlar om
%ska innehålla vettiga sökord
\title{Scaling up a GF grammar and preparing it for cool stuff for Swedish}
\author{Malin Ahlberg}
\maketitle
\newpage

\tableofcontents
\newpage

\abstract{
% This is the report of the work of expanding a GF grammar and preparing
% it for parsing of free Swedish. A lexicon consisting of more than 100 000 words
% has been extracted and we have constructed an automatic mapping between the
% annotation of the treebank Talbanken and GF. 
% The goal is to build a robust parser dealing with uncontrolled natural
% language, a parser which, for example, can parse all of Talbanken.
% Outside of this project, an equivalent GF project for English is being worked upon,
% as are theqniques for making the parser robust.
% With this in mind, we have extended the current GF grammar to adapt it for
% Swedish specifically so that it can parse the an extended set of basic
% constructions. We have completed a method for importing the eletronical 
% lexicon Saldo to a GF format and upprätat en connection mellan oss och Talbanken.

% sf dream to be able to work with npl. Parser important, statistics working
% on n-grams fail, closer to the human brain with rules? 
% Describe the basic.. of Swedish, in a format compatible with 20 other languages.
% example of sentence and how it can be parsed, stepping out of the controlled lang.
% Combine other resources, create free software.
% Equivalent for English and other languages coming.
% We aim for free parsing, this is preparation and up-scaling, finding difficulties.
}

\newpage

\section*{Acknowledgements}
% Ramona, Aarne, Elisabet, Markus, Peter?, Lars, Dan, Krasimir etc

\newpage
\chapter{Introduction}
\section{Introduction} 
Computers want to 'understand' language. Not good with flat strings.
We need to translate strings of natural language to a richer sturture,
a parse tree. So far there is no
freely availbe ruled-based parsing giving a deep analyse for Swedish.
%There are parsers which put only put part-of-speach tags to
%the text, and others that gives a deeper analyse. 
There are statistical parser, and parser which only are deal with controlled language.
The GF parser are one of those. 
%What is parsing, flat strings to trees, example picture, what is deep parsing.
So far stable and good when working with controlled languges, but recent experiments
have showed that it is also possible to parse free language.
To do this for Swedish, we need to  .. the resource grammar needed to covering
the language specific constructions. Further a lagre lexicon is also needed, as well
as a method for evaluation.
During this project, we have worked on and explored those things and some problems identified.
We have extended the Swedish resource grammar in GF, to give a better coverage and imported
a large lexicon. Work to compare the trees from Talbanken to GF and to translate them.


%What is controlled language,
%what is robust parsing and why we need it for free langugae, benefits of a grammar.
%GF is grammar formalism, many languages, many projects for controlled lang.
%GF grammar for Swedish, how big, not intended for parsing, but for building
%application grammars. Experiments of using the grammar for free parsing, methods
%being developped for this. Benefits of starting from GF. Free parsing is
%interesting. To do this, we need a better (bigger grammar), a large scale
%lexicon, and an evaluation method. This project has explored those things. The
%grammar also have other usages, important in itself, so we wanted to make it as
%covering as possible without being overallowing or too ambiguous.

Other work exist, but none is freely available. We want to create a open source
product, therefore can only use resources that allows this. Work can continue,
will not disappear, be made use of in other projects.
What we could not cover by this project, what have been covered before (in
the spring project), and why we choose those parts.

\section{Background}  
To build a parser or a grammar from scratch is a heavy work. The idea for this
project is to combine already existing parts into a bigger and more expressive grammar
as well as a large scale GF lexicon.
As all GF grammars, this one can be used for parsing, and we develop it by getting
examples, ideas and test material from a treebank.
The project is hence heavily dependening on three resources, which will be described
in the following sections.

\subsection{GF}
The main component of the the project is the Grammatical Framework\ref{gfbok} (GF). It is
a functional grammar formalism based on Martin-Löf type theory.
It has been developed for about 20? years and are inspired by programming langauges
like Haskell\footnote{}, ML\footnote{}, Prolog\footnote{} and Agda\footnote{}. Grammars written in GF can be used together with languages 
like Java, Java script and Haskell and works for both parsing and generation.

GF is developed with multilinguality in mind and makes an important distiction between
\textit{abstract} and \textit{concrete} syntaxes. The abstract is meant to be
shared by many languages, it models the semantics of the domain without blanda in
language specific features such as agreement or word order.

\begin{figure}[h]
\begin{verbatim}
 abstract TestGrammar {
  cat N ; V ; S ;

  fun 
    Pred : N -> V -> S ;
 }
\end{verbatim}
\caption{A small example of an abstract syntax}
\end{figure}

The example shows an abstract grammar defining three \textit{Categories},
one for nouns, one for verbs and one for sentences. An abstract grammar also gives
the types of functions. In this case we have \verb|Pred| which
tells us that by taking a noun and a verb we can form a sentence. It does not
give any information of how this is done, or how any of the categories should look.
maybe continue example with swedish and english, have DetNP : N -> N wich
gives 'the cat', 'katten'. Thereby introduce tables, parameters, fields.
Show abstract tree.

We now have two implementations of the abstract. The resulting GF grammar is able to
both parse a string to the abstract tree and to go in the other direction; to produce
a string of natural language given a abstract tree. This step is called linearization.
Translation is a consequence of this. Picture of sw. string -> tree -> eng. tree.

Valency information is normally built into the GF categories. Add V2 to the example.
Point out how N + V can be glued together to a string, the components cannot be seen
anymore.

A GF grammar usually covers a very restricted set of language, where few ambiguities
occur and much care can be taken to preserving the semantics between the different languages.
By using a controlled language like this,  good and reliable translation can be
ensured. Some examples of projects using GF are the math .. \cite{olga}, the
Online Syllogism Solver\cite{malin!}.  The framework is also used in the
european project MOLTO for online translation between upto 15 language.\\

One of the best... features of the framework is the resource libraries. Provided in the 
GF package, those gives a good start for all who want to write an application grammar.
Instead of letting the users reimplementing the morphological and the fundamental
syntactic structures of a language for each application, all this can be found
as libraries. So far, 20 languages are implemented all sharing the same abstract syntax.
They describe how to conjugate words, and how to construct phrases and sentences.
There are also a small test lexicon included, containing some hundered of common words.\\

As an open source-project, GF is constantely being developed and improved. New
languges are added, the compiler is being improved, ways of använad det på ett
lätt sätt are added and the possiblities to use GF in different environments
increased. There are  research about how to make better use of the dependent
types, for solving syllogism or generating natural language via Montague
semantics.
Further, experiments on using GF for free parsing has been conducted (Angelov,2010?).
This is also a future goal for our project and one of the delmål has been to
model an important part of Swedish in GF. The starting point was the resource
grammar and the result should still have a strong connection to this.

%Functor
%Main ideas, abstract and concrete, translation, parse- and abstract trees.
%Example of abstract.
%We say that this is a function that takes as argument and returns..
%Categories, valency built in (problems with ställa - på, under, vid). Parameters, fields.
%Example of a piece with Swedish, the reader gets familiar with how it looks.
%Also show how information 'disappears', become strings.
%What kind of linguistic analyse, what is the purpose to cover. Mention other uses,
%ideas of anaphores, montague? dependent types, ontologies, cool pictures by Krasse.
%
%The resources, why, how many, how to use, projects. functors.
%
%Still under big development, evolving, growing.
%For this project, only (mostly) important to model Swedish without abandoning the
%resources too much. Use of the big grammar for translation, why it might be hard with
%semantics and lexicon.
%Research about how to parse with gf by Krasimir, experiments with English.


\subsection{Swedish}
Swedish is a North-Germanic language spoken,
closely related to Norweigan and Danish. The languages share most of their
grammatical structures and are mutual intelligibility. Swedish is also one
a official language in Finland and approximately 10 million people speak Swedish.
Charasterictic for Swedish is the fact that it is a verb-second language: the
second constituent of a declarative main clause must consist of a verb.
The first constituent of the clause is usually made up of the subject,
although it likewise could consist of adverbial phrases or objects that are
being empathazied. \\

\enumsentence{
\begin{tabular}{lll}
  Johan & gick & sakta på gatan.\\
\emph{Johan} & \emph{walked} & \emph{slowely on the street.}\\
  Sakta & gick & Johan på gatan.\\
\emph{Slowly} & \emph{walked} & \emph{Johan on the street.}\\
  På gatan & gick & Johan sakta.\\
\emph{On the street} & \emph{walked}  & \emph{Johan slowly.}\\
\end{tabular}}

Inverted word order marks questions
\enumsentence{Gick Johan på gatan?\\
        \emph{Did Johan walk on the street?}}
The word order in subordinate clauses in also slightly different,
\enumsentence{Jag såg Anna. Hon hälsade \textbf{inte}.\\
        \emph{I saw Anna. She did not say hello.}\\
              Jag såg att Anna \textbf{inte} hälsade.\\
        \emph{I saw that Anna did not say hello.}}

Some other features typical for Swedish is it's use of passive, especially the
\textit{s-passive}, which is formed by adding an \emph{s} at the end of a verb
\enumsentence{Han jagade\textbf{s} av ett lejon.\\
        \emph{He was hunted by a lion.}}
Jfr.
\enumsentence{Ett lejon jagade honom.\\
        \emph{A lion was hunting him.}}

Constructions with 'det är/var' ('it is/was') is very common\cite{H&H309d}:
\enumsentence{Det var roligt att höra.\\
        \emph{I'm glad to hear that.}}
'Det' \emph{it} is also used as formal subject for intransitive verbs,
with the real subject put in the position of an object.
\enumsentence{Det står en älg på fältet.\\
        \emph{There is a mouse on the field.}}

The Scandinavian language have reflexive pronouns\cite{H&H310}
and special reflexive forms of the possessive pronouns for the 3rd
person\cite{H&H319}. Those are distinct from the normal 3rd person forms.
\enumsentence{
\begin{tabular}{ll}
Han slog sig. & Han såg sitt barn.\\
\vspace{3mm}
\emph{He hit him self.} & \emph{He saw his (own) child.}\\
Han slog honom. & Han såg hans barn.\\
\emph{He hit him (another person).}&\emph{He saw his (another person's) child.}
\end{tabular}}
The 1st and 2nd persons use the normal personal pronoun in object form.

%Basic info about Swedish. V2 lang, inverted word order and subordinate clauses.
%Passive, reflexive 'sitt', 'det är', 'det sitter en katt där', prepare the reader
%for what will be written in the grammar part.
%Adjectives but not verbs congruate with nouns. De är stora, They are big, Jag
%är stor, I am big
%Since the s-passive is the most common one in written Swedish \cite{SAG-34-1}, it has

\subsection{Talbanken}
For testing and evaluation of the grammar and lexicon, we needed to be able to
compare them against a reliable source.
Being an freely avaible, manually annotade, large scale treebank,
Talbanken\cite{talbanken} was perfect for this ändamål.
It was created in the 1970s at Lund University. In total it consists of four
parts. Two of them are transcribitions of spoken language, one a collection of
text written by high school (?) students, and the last one, section P,
consists of professionally written Swedish gathered from newspapers and ... .\\
Talbanken was also used to train the Swedish version of the Maltparser\cite{mal}
and was därmed reutgiven in a more modern version\cite{nivre}. This work
was done by Nivre and ... in 2005. They released it in some different XML-formats,
where the trees had been made deeper and smt more. \\
In this project, as well as for the Malt parser, the more than 6000 sentences
in section P has been used.
The treebank has served as an inspiration and an evaluation source throughout this
project. An automatic mapping between its trees and the abstract trees from GF has been
done, which will be explained in section \ref{sec:Mapping}.
%Differences in analyse, do we want a similar? 
%Uses of Talbanken in project, mapping to evaluate and test.  
%Look at output from Maltparser, could also comparing our parse trees to this.

\subsection{Saldo}
A good parser needs a good lexicon. We have used of Saldo\cite{saldo}, a
large electronic lexicon developed and maintained at Gothenburg University. It is
built upon Svenskt Associations Lexicon and contains information about modern
Swedish and the words are equipped with both sematic and a morphological
information. The user can find compounding analyses, examples of usage in corpora,
and graphs of sematically connected words.\\
Example of katt? \\
For our purpose however, only the morphological data was needed.\\
Example of morpho for katt.\\
Saldo provides this in XML format under LPGL. The data can be processed and translated to GF format as described in 
section \ref{sec:prog.saldo}.
Other usages?

\section{Related work}
Computational ling is active area for Swedish, there are many other Swedish parser,
but grammars?
Many of the parsers not ruled based, the most known one Malt ... .
However has been two, about CLE and what happend to it. Logic and translation
to Swedish, The spoken lang. translator.
Other Sw. parsers and grammars, evaluation of them? 
Other grammar frameworks.
% related and references work: CLE, Krasimirs book, CassSwe, GF book,SAG, gunlög,
% the multilingual treebank, Homse&Hinchcliff, Aarne i LILT

\chapter{Proceedings}
%\section{Proceedings med bättre ord}  
The purpose has been to prepare the grammar for large scale parsing.
Develop method used for English to fit Swedish, pynta the sw grammar with
language specific constructions and create methods for lexicon.
Earlier have done tool, but the lexicon was not big enough to be able
to fix it in this way. Tool may instead be used for complementing.

\section{Saldo}
\label{sec:prog.saldo}
In saldo there is also semantic info, so
We need the numbers from saldo if there are to lemmas with the same
identifier. sluta\_V and sluta2\_V (slöt). However gf does not want more than one table
if the forms are identical, saldo's morpho neither, but there are five in big saldo. 
We are just interested in morphology, so far. Dont want too big lexicon.

Saldo has been imported before (Krasimir, 2009?), but the code needed updates
to not miss many important words.

The basic algorith is the same. To understand this need to look at differences.
Example of gf and saldo table.
Explanation of the algorithm: map categories, try each paradigm, compare the table of forms
created. If several forms in saldo, gf should pick on of them. Saldo has forms for
compounds, gf does not. A grammar is written, the correct ones saved, the others 
retried. Try all paradigms which can be formed from saldo. 
There are a number of differences between and saldo Till skillnad från GF,
saldo does not contain forms that aren't used gf generates all forms. A form
needed to generate the gf table may not be given in saldo such as singular
forms for 'byxor'. Also contains idioms, 'hålla huvudet kallt' and has
different grammatical analyse, a lot of pronouns. Example. 

The new version have also been made more robust, print logs and list of all
words that wasn't imported together with a reason. Reson could be that it
cannot construct all forms or that the word class shouldn't be in the gf lexicon.
Costs very much memory, the saldo file is big, the generated lexicon is big.
Saldo is divided to save memory. Errors should just be reported, but if something oförutsett
happens and one part fail, the separation ensures that the others wont, also
possible to restart. Problems with naming convention, special
characters (å,ä,ö,apostrop,entrecôte.) needs to be translated , but avoid
classhes 'kaeltisk' 'kältisk', gf will crash by this, so we add a 1
in the end whenever we have changed a letter.

Other word classes we do not import and why.
Those should not be in the standard gf lexicon but maybe somewhere else.
The list can be used for adding other kinds of lexicons - idioms, have pos-tags.
Also to analyse what's wrong, or use the words with a word guesser. The code
has been adapted for this.
For exapmle pronouns - dets, have combined information from Talbanken to add them in the correct
category. How this works, look at tags. Small amount, can be done manually.

The saddest thing: no valency info. Crucial for GF. Only reflexive and verbs with particles
have any information. The earlier explained techniques may be used for this in Talbanken.
To get even bigger material, use Korp, but then no guarantee.

%see notes to add more about this, pronouns and the importing itself
%choices for adjectives vs. verbs for particips
The importing can be redone anytime.

\section{The grammar}
Although we can't expect to get a full coverage, we would like a tillfredsställande
Explain what's in the resources, constructions present in most languages. 
Been working on a grammar fragment based on Talbanken,
like cool people like Montague. May also be interesting for other purposes.
Swedish shares with Nor and Dan. Extra module, covering things like sw is a
v2-lang, focusing parts of sentences. 



\subsection{GF for Swedish}
The GF resource grammars contains a fundamental description of Swedish,
covering the morphology, many phrases and clauses.
Covers sentences like
han har inte ätit två gula äpplen idag
word order, agreement, tempus, adverbs, adjectivs...
and relatives
han såg katten han tyckte om
Also info in the Extra module, where more language specific constructions
can be found.
honom ville hon inte tänka på.

Before describing what has been added, will give a short introductions to the
the cats of resource grammar and the sw ipmlementation in particular.

Most important now is NP and VP.
Starting from N, get to CN what is it, what can we do now.
To turn in into a NP need to fix determination, by Det. Dets have .. fixed,
but allow variantion in.. 
What is the difference? Some words on differences for pronouns.
Example, use vartiants {}.
There are also Quants, which .... 
Once we have a NP, can use Predets. What they are.

For verbs, but also others there is an important distinction for valency.
Either directly to VP if V, or needs to be given arguments. Can then add adverbs.
Maybe talk about ställa.

%Besides the categories NP,VP,AP there are also the import Det and Quant, Predet.

%What was in the resources
%description of the existing grammar, the types, the possibilities
%VP, NP.
%valency built in (problems with ställa - på, under, vid).
%Determiners, quants, predets, The difference from how pronouns are regarded. 
%Variants, nonexist.
%coverage of standard Swedish.

\subsection{Added}
Summary of what have been added and motivation.

kommer att - for scand, extra tempus, example
example table

%%PASSIV
Is passive also more used in Swedish?
There are two ways of forming passive verb phrases in Swedish: the 
periphrastic passive, formed by using a modale auxiliary verb 
\enumsentence{Bilden blev tagen av  av rymdsonden Galileo}
\enumsentence{The picture was taken by spacecraft Galileo.}
or by adding an \emph{s} to
the verb:
\enumsentence{Bilden togs av rymdsonden Galileo}
\enumsentence{The picture [took+s] by spacecraft Galileo.}
This second variant, the s-passive, is commonly used in Swedish, some research
suggest that it is used in more than 80\% of the times \cite{laanemets}.
The other scandinavian languages prefers the periphrastic passive, especially in spoken langauge.
\enumsentence{Bildet ble tatt av romsonden (NO)}
The resource grammar for Scandinavian therefore allowed/expressed constructions using auxiliary verb. \\
\verb|PassV2 : V2  -> VP | \\
\verb|         äta -> blev äten| \\
This rules allows two-place verbs to be used in passive by using \emph{bli} (\emph{become}), and thereby
turned into comlpete verb phrases; they no longer need an object.
The s-passive has now been added as the standard case for the Swedish GF
grammar. Periphrastics passive is still allowed, but as an alternative rather
than the default.
The grammar also allows using three-place verbs \verb|V3| , as \emph{give} to
form passives:
\enumsentence{Betyget 3 ges till många elever}
\label{ex:passV3}
by letting the first object be `removed' and given by the subject. 
Sentences where the second object is put first in the sentence, like 
\enumsentence{Till många elever ges betyget 3}
is a variant of example \ref{ex:passV3}, \emph{`många elever'} is still the object but put in focus. The grammar
parse this by applying \verb|FocObj|.
%%

%% Clauses
\subsection{The clause-level}
The resource grammar already provided a number of ways to express relative and embedded clauses, such as
\enumsentence{Frågan är vilken färg den har\\
                   Pojken, som är blyg, tystnar\\
                   Han såg kunden som tyckte om sallad\\
                   Jag tänkte på huset vilket hon bodde i} % better example!!

Since those constructions are inspired by other languages, some of them may sound quite
non-natural in Swedish, such as the \verb|RelS|. This combines a sentence \emph{`Hon sover'}
and a relative sentence \emph{`som/vilket är bra'} which wrongly expressed 
\emph{`She sleeps, which is good'} as
\emph{`Hon sover som är bra'}. The error comes from the fact that \emph{`som'} is
embedded in the relative clause. To fix this, an extra parameter was needed to tell whether
\emph{`som'} or \emph{`vilket'} should be used. \\
Another complication stems from \verb|RelCl| which express the english \emph{`such that'}.
The Swedish version \emph{`sådan att'} sounds stelbent, when used outside of logic books.
\enumsentence{Jag vill ha en katt sådan att den inte fäller.} 
An more natural sounding alternative would be
\enumsentence{Jag vill ha en sådan katt som inte fäller.}
and hopefully this will be implemented soon, otherwise: why is it hard and how we can say
'en katt sådan att det regnar'. It is relative but has on empty field, still congruation is
needed.\\


Formal subjects \cite{SAG-19,H&H-309d} is often used in Swedish
\enumsentence{Det sitter en fågel på taket}
\emph{`There is a bird sitting on the roof'}
or, literally
\emph{`It sits a bird on the roof'}.\\
\emph{`Det'} has the position of the subject, and the real subject, 
\emph{`en fågel'} the one of an object.
Transitive verbs may not be used like this
\enumsentence{*Det äter en fågel på taket.}
unless their in passive form
\enumsentence{Det äts en fågel på taket.}
%det finns/saknas/fattas
There are also restrictions on the nounphrase of the subject, it can only
be determined in certain ways
\enumsentence{*Det sitter den fågel på taket. \\
              **Det sitter fåglarna på taket.}
The distinction lays in the definitness of the noun itself, (not the whole noun phrase..)
it must be indefinite, even though determiners are allowed. Jfr
\enumsentence{Det sitter några fåglar på taket\\
              **Det sitter vardera fåglarna på taket}
%%Quantifiers like \emph{`denna'} or determiners like \emph{`samtliga'} may
%%not be used, but the determiners \emph{`många'}, \emph{`några'} and the quantifier
%%\emph{`ingen'} work well.
%%Semantic difference, atm we allow any NP. \\
The grammar handles this by looking at the determiner used
\verb|FormalSub : SimpleVP -> Det -> NP -> Cl ;|
The function combines a verb phrase a determiner and a noun phrase in to a clause.
\verb|SimpleVP| is a normal VP, but with the restriction that the main verb must be
intransitiv, a \verb|V|, (and passives?). If the  determiner requires a noun in determinened
form, the clause is put to NONEXIST.

'det är kul att..' Näst vanligaste två-ordskombinationen i talad svenska.
%%
%embedded - relative, 'sådan att' not really nice for normal swedish, and no agreement
%between subject 'en katt sådan att det regnar'.
pronouns/Quant. Many, created categories for 'sådan', and 'fler'. Not many in each,
has to decide, one category for each?
Genitive -problem 1 cannot be alone (fixed) 2 some words have id. (no) gen form
           fixing this include restructuring all
Det-utr. not effect for the some Dets, like 'sådan', no difference between utrum and neutrum.
 Can vary in number, since is a NP. Do not want to cut connection to abstract but changing them,
 so added one more. Leads to ambiguities in some cases, but considering that we would want to use
 this for translation, neutr and utr may differ in the other language.

-- or is this really just AdVs? 
to change position of adv:
normal: after verb 'han äpplet redan'
now want: 'äter redan äpplet'
          'har redan ätit äpplet' 
          'har ätit redan äpplet' :( which doesn't work either :)

but have other words like 'bara' which can be put in other positions.
  but needs to restruct whole of VP, since no field is befor fin.
  Can also act before the noun or as AdV. Examples.
\verb|s SPres Simul Neg Main : han bara sover inte alltid|
  have added a field in VP, a0, show table.
  makes some weird constructions when using modal verbs, 
  \verb|s SPres Anter Pos Main : han bara har alltid sovit|
  but rather than changing to 'han har bara sovit', we keep it. Is a possible
  tolkning and the other case is covered by using it as an AdV, we don't want
  ambiguities.

Also needed categories for 'pronouns' wich can act in different ways.
and 'antal','sorter' - need indetermined, antal plural, sorter either.
Should be able to modify by a godtyckligt antal adjective, and they must be determined.
'(ett stort antal) människor' and not 'ett stort (antal människor)'
Don't want them as a NPs or CNs since we don't want other Ns to be used this way
'en flicka kaffe', this is then Appos? (Or do we?). Should also not be utsatta for what other
NPs or CN can be. Dessutom, we want to make use of the comp-fiel. 'mamma till', but is this really
the same??

The VP require many changes, to add passive (no), PPart, bara.
About PPart, but do we need it with new Saldo? 'de är stuckna' , 'den nyligen funna'
for the last one, we need VPSlash to allow adverb.


%% Refl 
Could varandra be thought of as reflexive for 'oss'?
varandra and sin/sitt/sig. Not subject, but NP.
Want the following:
\enumsentence{Han såg sina barns skor.}
\enumsentence{Sina vantar hade han glömt på tåget.}
but not
\enumsentence{*Sina vantar var kvar på tåget.}
\enumsentence{Han ger sina pengar till sina syskon.}
\enumsentence{*Jag ger sina pengar till sina syskon.}
\enumsentence{Han är längre än sin kompis.}
\enumsentence{Hon tyckte om alla sina elever.}
\enumsentence{Han såg sin hund och sitt hus}
\label{ex:refGen}
\enumsentence{Han är längre än sin kompis}
\label{ex:ReflAP}
% hon ber sin katt gå - sin subject in clause.

Suggest a connection between the subject and the object.
Could be combined with 'sig', a ReflNP without another noun.
-- man ser sig!!
Make a distinction between objects and subjects?
do not want to copy all rules for another category. Would like depent types!
work pretty well by letting all NP depend on the subject in the phrase.
The current version is nice and works for recursive genitivs!
Extra info, but maybe this is just the case about NPs.
Discuss how other solutions, NPR and Predet : Predet -> CN -> NPR ;
*Slash* : VPSlash -> NPR -> VP ;
Keep number of fields down, but introduces another category, and one more
label to the parse tree, is it interesting? How to combine it with genitive 
without making it ambiguous (Han såg barnens hund).
What if we add alot of other rules that uses NP, will they all need to be
duplicated then?

'han såg sin mammas bil', they same could be applied to
APs for example \ref{ex:ReflAP} or Adv for
\enumsentence{Han leker oftare än sin kompis}
 (ComparAdvAdj)
Atm you can't focus this kind of object, then a ClSlash need info about the
subject. FunRP also tricky (and what does it mean, we can say
Han såg sin katt i vilken musen låg), in this case RP needs to give more information.
Kort sagt, the information about the subject must be spread
in different parts of the sentence. 
Could also decide to always produc 'sin' -> jag såg sin bil
to avoid the dependence, but not nice. Or keep a flag in NP saying if it is
in third person, and add nonexist for others. But this way we can also get rid
of ReflVP and the possessive prons \verb|PossPron| should only be used for somebody else:
'jag såg din bil'.
Unfortunaly, have to change ReflVP, this only allows this in object for VPSlash, not FocObj,
not 'alla sina', usw. Want it to handle conjuction as in example \ref{ex:refGen}
etc.
%%

In general, many categories need more parameters and fields, tex Quant can be used as NP
and therefor needs to depend on Case, de -> dem.

here or in robust: SupCl, det as VP or VV or S for S (remember Foc)

\subsection{Testing}
Hard do test, can't verify. Regression testing. Talbanken is still too hard.
While developing: regression, 150 sentences to start with. Made longer, both good and bad,
to see how many ways things can be parsed.
Test how many parse trees that are generated.
Look at tables for a VP tex.
Some results.

% One part about the grammar. The state of the original grammar, what was covered and not. 
% What parts that I have been working on and why, 
% what constructions I wanted to add, examples of sentences that should work
% and of sentences that shouldn't be covered. How to separate Swedish from
% Scandinavian, bigger reconstructions in the grammar. 
% sammanhängande explanation of the swedish grammar, choices I have made, why
% regression testing

\section{Mapping}
\label{sec:Mapping}
To connect GF with other annotations, to evaluate. Identify missing constructs,
missing words. More reasons from presentation.
Similar for English and Penntree, but PennTree and Talbanken different, plus swedish
makes a different. More possible word orders (or at least more commonly used).
One deep and one flatter version of Talbanken, deep contains ca 5 more tags.
The flat was used since the deep one
isn't deep in the same way as GF. Example with CNP (s5). Some
more info about VPs are given, but do not think this will give any relevant information.
Talbanken also contains frequent .. tags XP, XX osv. We cannot expect to translate
this to anything but metas.
How we try to build up a 'parse tree', a S with a NP VP for example. 
How verbs look for complements (or inverted order), how metas are added.
Description of most important tags from Talbanken, their translation.
Description of the ones most hard to translate (particles, flat sentences)
Example of nice translation and one with many metas, and one that doesn't work.
Developed by adding one mapping for each tag, and by examples.
How we cannot cover all cases. The testing: regression testing again,
very easy to destroy what was working before. 
How we need valency information.
If we don't take the verb valency into account, allow all verbs to have Gf category V,
we get a slightly better result,
from 62.5\% to 65.4\%.  (see EvalText). The complemnt of the verbs should tell which category
they are anyway.
Evaluation. -> Works for both flat and deep, but with sligthly better results for the flat one.

% hard examples like
% ger sig till henne -> ReflSlash ge henne in gf
% questions with objects that should be moved



%One part about the mapping from Talbanken to GF, description. About the purpose.
%This part explains
%the tags in Talbanken and gives examples of how they are translated,
%examples of what can be parsed and what cannot.
%about how to make the code work for swedish, differences
%regression testing


\chapter{Summary}
\section{Results and discussion}
Results from mapping, can probably be improved, bit by bit, depends much on words.
Valencys would therefor help.  
Saldo, made a big difference to renew the lexicon. Tested what was missing now from talbanken,
results. Mostly compound nouns, which we can't expect in the dictionary.

Grammar - hard to evaluate automatically, but Elibet is an expert who has been involved
in the process to verify the solutions. Testing trees against talbanken.
The big lexicon makes it very slow, eats all memory

Discussion of the results and methods, why are the results good, why are they not better.
What other methods could have been used? What did we/I expect and what happend?


\section{Future Work}
What is still left to do? Description and direction of future work:
how to make the parser robust, elipses and long sentences, names and numbers.
What is still needed in the grammar, how important that is. 
how to get valency information 'sitter och läser', description of extracting this
or handling it by the robust parser or the grammar itself.
make the parser faster by using smaller lexicon.
Lexicon tool on the fly.
Getting and using probabilities, to disambiguate.
process list from saldo.
Use multiligual treebank?
stop ambiguities by fixing so that Gen of a quant isn't a quant, or better, that all fields
have a genitive field.
More fields in the grammar? Or leave it all to the robust parser


\section{Conclusion}
There is obviously a lot do still, will never get finish with NPL.
Grammar may need even bigger changes, to allow enough but not too much.
Different from writing a grammar for generation.
Results from mapping shows that the translation is doable, but becomes harder
since talbanken don't have formal rules.
The saldo shows how easy to make use of utomstående resources, promising if
we want to extract other information. 
Hard, but not finished! 


\end{document}
