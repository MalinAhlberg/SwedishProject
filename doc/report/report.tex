\documentclass{report} 
\usepackage[utf8]{inputenc}
\usepackage{graphicx}
\usepackage[margin=1.5in]{geometry}

\begin{document}
% ska gå att förstå och locka till läsning
%ska tydligt beskriva vad rapporten handlar om
%ska innehålla vettiga sökord
\title{Scaling up a GF grammar and prepare it for cool stuff for Swedish}
\author{Malin Ahlberg}
\maketitle
\newpage

\tableofcontents
\newpage

\abstract{
% sf dream to be able to work with npl. Parser important, statistics working
% on n-grams fail, closer to the human brain with rules? 
% Describe the basic.. of Swedish, in a format compatible with 20 other languages.
% example of sentence and how it can be parsed, stepping out of the controlled lang.
% Combine other resources, create free software.
% Equivalent for English and other languages coming.
% We aim for free parsing, this is preparation and up-scaling, finding difficulties.
}

\newpage

\section*{Acknowledgements}
% Ramona, Aarne, Elisabet, Markus, Peter?, Lars, Dan, Krasimir etc

\newpage
\chapter{Introduction}
\section{Introduction} 
What is parsing, flat strings to trees, what is deep parsing. What is controlled language,
what is robust parsing and why we need it for free langugae, benefits of a grammar.
GF is grammar formalism, many languages, many projects for controlled lang.
GF grammar for Swedish, how big, not intended for parsing, but for building
application grammars. Experiments of using the grammar for free parsing, methods
being developped for this. Benefits of starting from GF. Free parsing is
interesting. To do this, we need a better (bigger grammar), a large scale
lexicon, and an evaluation method. This project has explored those things. The
grammar also have other usages, important in itself, so we wanted to make it as
covering as possible without being overallowing or too ambiguous.

Other work exist, but none is freely available. We want to create a open source
product, therefor can only use resources that allows this. Work can continue,
will not disappear, be made use of in other projects.
What we could not cover by this project, what have been covered before (in
the spring project), and why we choose those parts.

\section{Background}  
Too long way to build a parser/grammar from scratch. Idea to combine existing parts. 
The project is heavily dependening on three resources, will describe them here.

\subsection{GF}
Most important part. Been developed for ? years, inspiration from Haskell, ML,
and also Prolog, Agda. Can be used with other programming langauges. 
Main ideas, abstract and concrete, translation, parse- and abstract trees.
Categories, valency built in. Parameters, fields.
What kind of linguistic analyse, what is the purpose to cover. Mention other uses,
ideas of anaphores, montague? dependent types, ontologies.
For this project, only (mostly) important to model Swedish without abandoning the
resources too much. Use of the big grammar for translation, why it might be hard with
semantics and lexicon.
Research about how to parse with gf by Krasimir, experiments with English.
Example of a piece with Swedish, the reader gets familiar with how it looks.
Also show how information 'disappears', become strings.

\subsection{Swedish}
Basic info about Swedish. V2 lang, inverted word order and subordinate clauses.
Passive, reflexive 'sitt', 'det är', 'det sitter en katt där', prepare the reader
for what will be written in the grammar part.


\subsection{Talbanken}
For evaluating and testing, comparing our parser we needed a reliable source.
Talbanken manually anotade by ... in .. . Consists of four parts : ... . 
Also used by Maltparser and was developed in 2005 by Nivre, deeper trees.
Different parts (written, spoken) and deep/flat. Differences in mapping
on flat/deep?
Differences in analyse, do we want a similar? 
Uses of Talbanken in project, mapping to evaluate and test.  
Look at output from Maltparser, could also comparing our parse trees to this.

\subsection{Saldo}
The last(..) part is a big lexicon. The electronical Saldo is developped at GU.
Built from sal,
Contains much more information,
graph, word relationships, semantics, but also morphological info.
Example of a word, how the info looks.
About size and being maintained. Other usages?
Large, freely available and compatible with GF. 

\section{Related work}
Computational ling is active area for Swedish, there are many other Swedish parser,
but grammars?
Many of the parsers not ruled based, the most known one Malt ... .
However has been two, about CLE and what happend to it. Logic and translation
to Swedish, The spoken lang. translator.
Other Sw. parsers and grammars, evaluation of them? 
Other grammar frameworks.
% related and references work: CLE, Krasimirs book, CassSwe, GF book,SAG, gunlög,
% the multilingual treebank, Homse&Hinchcliff, Aarne i LILT

\chapter{Proceedings}
%\section{Proceedings med bättre ord}  
 The purpose has been to prepare the grammar for large scale parsing.
 Develop method used for English to fit Swedish, pynta the sw grammar with
 language specific constructions and create methods for lexicon.
 Earlier have done tool, but the lexicon was not big enough to be able
 to fix it in this way. Tool may instead be used for complementing.

\section{Saldo}
In saldo there is also semantic info, so
We need the numbers from saldo if there are to lemmas with the same
identifier. sluta\_V and sluta2\_V (slöt). However gf does not want more than one table
if the forms are identical, saldo's morpho neither, but there are five in big saldo. 
We are just interested in morphology, so far. Dont want too big lexicon.

Saldo has been imported before, ref, but the code needed updates to not miss
many important words.

The basic algorith is the same. To understand this need to look at differences.
Example of gf and saldo table.
Explanation of the algorithm: map categories, try each paradigm, compare the table of forms
created. If several forms in saldo, gf should pick on of them. Saldo has forms for
compounds, gf does not. A grammar is written, the correct ones saved, the others 
retried. Try all paradigms which can be formed from saldo. 
There are a number of differences between and saldo Till skillnad från GF,
saldo does not contain forms that aren't used gf generates all forms. A form
needed to generate the gf table may not be given in saldo such as singular
forms for 'byxor'. Also contains idioms, 'hålla huvudet kallt' and has
different grammatical analyse, a lot of pronouns. Example. 

The new version have also been made more robust, print logs and list of all
words that wasn't imported together with a reason. Reson could be that it
cannot construct all forms or that the word class shouldn't be in the gf lexicon.
Costs very much memory, the saldo file is big, the generated lexicon is big.
Saldo is divided to save memory. Errors should just be reported, but if something oförutsett
happens and one part fail, the separation ensures that the others wont, also
possible to restart. Problems with naming convention, special
characters (å,ä,ö,apostrop,entrecôte.) needs to be translated , but avoid
classhes 'kaeltisk' 'kältisk', gf will crash by this, so we add a 1
in the end whenever we have changed a letter.

Other word classes we do not import and why.
Those should not be in the standard gf lexicon but maybe somewhere else.
The list can be used for adding other kinds of lexicons - idioms, have pos-tags.
Also to analyse what's wrong, or use the words with a word guesser. The code
has been adapted for this.
For exapmle pronouns - dets, have combined information from Talbanken to add them in the correct
category. How this works, look at tags. Small amount, can be done manually.

The saddest thing: no valency info. Crucial for GF. Only reflexive and verbs with particles
have any information. The earlier explained techniques may be used for this in Talbanken.
To get even bigger material, use Korp, but then no guarantee.

%see notes to add more about this, pronouns and the importing itself
%choices for adjectives vs. verbs for particips
The importing can be redone anytime.

\section{The grammar}
Although we can't expect to get a full coverage, we would like a tillfredsställande
coverage of standard Swedish. Been working on a grammar fragment based on Talbanken,
like cool people like Montague. May also be interesting for other purposes.
Explain what's in the resources, constructions present in most languages. 
Swedish shares with Nor and Dan. Extra module, covering things like sw is a
v2-lang, focusing parts of sentences. Sw also makes use of passive s-form a lot.
kommer att - for scand, extra tempus, example
passiv s: but wich verbs can be made passive? 
embedded - relative, 'sådan att' not really nice for normal swedish, and no agreement
between subject 'en katt sådan att det regnar'.
pronouns/Quant. Many, created categories for 'sådan', and 'fler'. Not many in each,
has to decide, one category for each?
Genitive -problem 1 cannot be alone (fixed) 2 some words have id. (no) gen form
           fixing this include restructuring all
Det-utr. not effect for the some Dets, like 'sådan', no difference between utrum and neutrum.
 Can vary in number, since is a NP. Do not want to cut connection to abstract but changing them,
 so added one more. Leads to ambiguities in some cases, but considering that we would want to use
 this for translation, neutr and utr may differ in the other language.

-- or is this really just AdVs? 
to change position of adv:
normal: after verb 'han äpplet redan'
now want: 'äter redan äpplet'
          'har redan ätit äpplet' 
          'har ätit redan äpplet' :( which doesn't work either

but have other words like 'bara' which can be put in other positions.
  but needs to restruct whole of VP, since no field is befor fin.
  Can also act before the noun or as AdV. Examples.
\verb|s SPres Simul Neg Main : han bara sover inte alltid|
  have added a field in VP, a0, show table.
  makes some weird constructions when using modal verbs, 
  \verb|s SPres Anter Pos Main : han bara har alltid sovit|
  but rather than changing to 'han har bara sovit', we keep it. Is a possible
  tolkning and the other case is covered by using it as an AdV, we don't want
  ambiguities.

Also needed categories for 'pronouns' wich can act in different ways.
and 'antal','sorter' - need indetermined, antal plural, sorter either.
Should be able to modify by a godtyckligt antal adjective, and they must be determined.
'(ett stort antal) människor' and not 'ett stort (antal människor)'
Don't want them as a NPs or CNs since we don't want other Ns to be used this way
'en flicka kaffe', this is then Appos? (Or do we?). Should also not be utsatta for what other
NPs or CN can be. Dessutom, we want to make use of the comp-fiel. 'mamma till', but is this really
the same??

The VP require many changes, to add passive (no), PPart, bara.
About PPart, but do we need it with new Saldo? 'de är stuckna' , 'den nyligen funna'
for the last one, we need VPSlash to allow adverb.

varandra and sin/sitt/sig. Not subject, but NP.
do not want to copy all rules for another category. Would like depent types!
work pretty well by letting all NP depend on the subject in the phrase.
The current version is nice and works for recursive genitivs!
Extra info, but maybe this is just the case about NPs.
'han såg sin mammas bil', they same could be applied to
APs for 'han är längre än sin kompis' or Adv for 'han leker oftare än sin
kompis'. (ComparAdvAdj)
Atm you can't focus this kind of object, then a ClSlash need info about the
subject. FunRP also tricky, in this case RP needs to give more information.
Kort sagt, the information about the subject must be spread
in different parts of the sentence. 
Could also decide to always produc 'sin' -> jag såg sin bil
to avoid the dependence, but not nice. Or keep a flag in NP saying if it is
in third person, and add nonexist for others. But this way we can also get rid
of ReflVP and the possessive prons \verb|PossPron| should only be used for somebody else:
'jag såg din bil'.

In general, many categories need more parameters and fields, tex Quant can be used as NP
and therefor needs to depend on Case, de -> dem.

here or in robust: SupCl, det as VP or VV or S for S (remember Foc)

\subsection{Testing}
Hard do test, can't verify. Regression testing. Talbanken is still too hard.
While developing: regression, 150 sentences to start with. Made longer, both good and bad,
to see how many ways things can be parsed.
Test how many parse trees that are generated.
Some results.

% One part about the grammar. The state of the original grammar, what was covered and not. 
% What parts that I have been working on and why, 
% what constructions I wanted to add, examples of sentences that should work
% and of sentences that shouldn't be covered. How to separate Swedish from
% Scandinavian, bigger reconstructions in the grammar. 
% sammanhängande explanation of the swedish grammar, choices I have made, why
% regression testing

\section{Mapping}
To connect GF with other annotations, to evaluate. Identify missing constructs,
missing words. More reasons from presentation.
Similar for English and Penntree, but PennTree and Talbanken different, plus swedish
makes a different. More possible word orders (or at least more commonly used).
One deep and one flatter version of Talbanken, deep contains ca 5 more tags.
The flat was used since the deep one
isn't deep in the same way as GF. Example with CNP (s5). Some
more info about VPs are given, but do not think this will give any relevant information.
Talbanken also contains frequent .. tags XP, XX osv. We cannot expect to translate
this to anything but metas.
How we try to build up a 'parse tree', a S with a NP VP for example. 
How verbs look for complements (or inverted order), how metas are added.
Description of most important tags from Talbanken, their translation.
Description of the ones most hard to translate (particles, flat sentences)
Example of nice translation and one with many metas, and one that doesn't work.
Developed by adding one mapping for each tag, and by examples.
How we cannot cover all cases. The testing: regression testing again,
very easy to destroy what was working before. 
How we need valency information.
If we don't take the verb valency into account, allow all verbs to have Gf category V,
we get a slightly better result,
from 62.5\% to 65.4\%.  (see EvalText). The complemnt of the verbs should tell which category
they are anyway.
Evaluation. 

%One part about the mapping from Talbanken to GF, description. About the purpose.
%This part explains
%the tags in Talbanken and gives examples of how they are translated,
%examples of what can be parsed and what cannot.
%about how to make the code work for swedish, differences
%regression testing


\chapter{Summary}
\section{Results and discussion}
Results from mapping, can probably be improved, bit by bit, depends much on words.
Valencys would therefor help.  
Saldo, made a big difference to renew the lexicon. Tested what was missing now from talbanken,
results. Mostly compound nouns, which we can't expect in the dictionary.

Grammar - hard to evaluate automatically, but Elibet is an expert who has been involved
in the process to verify the solutions. Testing trees against talbanken.
The big lexicon makes it very slow, eats all memory

Discussion of the results and methods, why are the results good, why are they not better.
What other methods could have been used? What did we/I expect and what happend?


\section{Future Work}
What is still left to do? Description and direction of future work:
how to make the parser robust, elipses and long sentences, names and numbers.
What is still needed in the grammar, how important that is. 
how to get valency information 'sitter och läser', description of extracting this
or handling it by the robust parser or the grammar itself.
make the parser faster by using smaller lexicon.
Lexicon tool on the fly.
Getting and using probabilities, to disambiguate.
process list from saldo.
Use multiligual treebank?
stop ambiguities by fixing so that Gen of a quant isn't a quant, or better, that all fields
have a genitive field.


\section{Conclusion}
There is obviously a lot do still, will never get finish with NPL.
Grammar may need even bigger changes, to allow enough but not too much.
Different from writing a grammar for generation.
Results from mapping shows that the translation is doable, but becomes harder
since talbanken don't have formal rules.
The saldo shows how easy to make use of utomstående resources, promising if
we want to extract other information. 
Hard, but not finished! 


\end{document}
