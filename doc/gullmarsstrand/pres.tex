\documentclass[10pt]{beamer} 
%\usefonttheme{structuresmallcapsserif} 
%% \usepackage{beamerthemeshadow}
\usepackage{verbatim} 
%% \usetheme{Pittsburgh}
\usepackage{colortbl}
\usepackage{graphicx}
\usepackage{tabularx}
\usepackage[utf8]{inputenc}
\usepackage{listings}
\usepackage{cancel}
 \renewcommand{\baselinestretch}{1.5}
     \normalsize
     
% THIS SHOULD BE HERE!
% No unimportant, irrelevant things. Only information.
% Only code if it is of significance.

% get inspired by:
% Simon Jones, Microsoft research, videos.
% John Hughes article How to give a good research presentation.
\title{A Wide-Coverage Grammar and Parser for Swedish}
\subtitle{\large First results and perspectives}
\author{Malin Ahlberg \\ Gothenburg University}
\date{}

\begin{document}
\maketitle

\begin{frame}
    \framesubtitle{The goal}
    Malin Ahlberg \\
 \pause 
    Gothenburg University
\end{frame}

\begin{frame}
\frametitle{What I'm doing now}
\framesubtitle{} 
mapping
grammar 
saldo
\end{frame}

\begin{frame}
\frametitle{mapping}
\framesubtitle{Example} 
Nice example
\end{frame}


\begin{frame}
\frametitle{mapping}
\framesubtitle{Why, how} 
why it's hard
why it's good
goals
\end{frame}

\begin{frame}
\frametitle{mapping}
\framesubtitle{problems} 
so far 42\% of the easy sentences are totally mapped 
       12\% not at all

avp
differnt parsings
\end{frame}

\begin{frame}
\frametitle{grammar added}
\framesubtitle{examples} 
Easy sentences with known words (old testsuite) : 90 out of 118 (76\%) 
PassVP "äpplet åts"
  ("äpplet blev ätet")


ReflNP, ReflSlash
  relate to Peter - sig/sig själv, can't say hon ger den till sig
  (solved?)

RelNP' "flickan, som inte ätit äpplen"
 (RelNp "flickan , sådan att hon inte åt äpplen")

AdvVPSlash "hon åt redan äpplet" (hon åt äpplet redan")

PPartAP "det är sett","det ätna äpplet"
 (not "det är givet till henne" osv, "minskad efterfrågan")
  

list of functions
examples
difficulties
\end{frame}



\begin{frame}
\frametitle{Particip}
\framesubtitle{What type does a particip have?} 
attribut- predikativ
en ökad effektiv information (ökat vanligt!!)
en minskad benägenhet
i en gången tids svenska bondesamhälle  4872    
en fritt vuxen natur
centralt belägna, eftertraktade lägenheter
de stegrade hyrorna
fyra representerade kyrkor
en kvalificerad majoritet
misstänkt (av)
kokt potatis (av)
som aldrig är skalad (industriskalad, skalkokt)
vår behandlade potatis

är oroad
beprövad (av)
de levererar den doppad i ...
\end{frame}

\begin{frame}
\frametitle{ReflGenVP}
\framesubtitle{Problem} 
sin egen lilla stubbe -ok 
alla sina barn
all vår undervisning
sin urgamla rätt att... 
hur löses det för vår? (inget problem för det hör ej ihop med subj)

\end{frame}

\begin{frame}
\frametitle{Goals}
\framesubtitle{Goals for christmas} 
finish and evaluate mapping
extend grammar with: relative clauses, embedded clauses
new mapping of saldo - making it easy to redo
evaluation
\end{frame}

\begin{frame}
    \frametitle{The end}
\huge\begin{center}Questions?\end{center}
\end{frame}
\end{comment}
\end{document}

