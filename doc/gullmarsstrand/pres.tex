\documentclass[10pt]{beamer} 
%\usefonttheme{structuresmallcapsserif} 
%% \usepackage{beamerthemeshadow}
\usepackage{verbatim} 
%% \usetheme{Pittsburgh}
\usepackage{colortbl}
\usepackage{graphicx}
\usepackage{tabularx}
\usepackage[utf8]{inputenc}
\usepackage{listings}
\usepackage{cancel}
 \renewcommand{\baselinestretch}{1.5}
     \normalsize
     
% THIS SHOULD BE HERE!
% No unimportant, irrelevant things. Only information.
% Only code if it is of significance.

% get inspired by:
% Simon Jones, Microsoft research, videos.
% John Hughes article How to give a good research presentation.
\title{A Wide-Coverage Grammar and Parser for Swedish}
\subtitle{\large First results and perspectives}
\author{Malin Ahlberg \\ Gothenburg University}
\date{}

\begin{document}
\maketitle

\begin{frame}
\frametitle{Current work}
\framesubtitle{} 
A mapping of Talbanken05 trees to GF trees\\
Extending the grammar \\
Re-import saldo\\
\end{frame}

\begin{frame}
\frametitle{Mapping}
\framesubtitle{Example} 
Nice example
\end{frame}


\begin{frame}
\frametitle{Mapping}
\framesubtitle{Why?} 
In order to \\
\begin{itemize}
\item{evaluate the parser}
\item{identify grammatical constructions not current in the GF grammar}
\item{enable good lexical extraction}
\item{extract probabilites for GF functions}
\end{itemize}
% why it's hard (396 hur och hur)
%goals
\end{frame}

%\begin{frame}
%\frametitle{mapping}
%\framesubtitle{problems} 
%so far 42\% of the easy sentences are totally mapped 
%       10\% not at all
%
%avp
%differnt parsings
%\end{frame}

\begin{frame}
\frametitle{The grammar}
\framesubtitle{Current work} 
\begin{tabular}{ll}
\textbf{PassVP} & V2 $\rightarrow$ VP (should be VPSlash $\rightarrow$ VP) \\
& \emph{äpplet åts},\emph{boken gavs till honom} \\ % \;
 %\emph{("äpplet blev ätet")}\\ % boken gavs till honom

\textbf{ReflSlash} & VPSlash $\rightarrow$ VP \\
& \emph{han ger sina små barn till dem} \\
% \\ but not \emph{sig själv}, \emph{han äter alla sina äpplen} or \emph{han ger sig till sig}  \\
  %relate to Peter - sig/sig själv, can't say hon ger sig till sig
%
%%RelNP' "flickan, som inte ätit äpplen"
%% (RelNp "flickan , sådan att hon inte åt äpplen")

% CompAdv here, plus we need ComlVVAdv, Pass? osv
\textbf{AdvVPSlash} & VPSlash $\rightarrow$ Adv $\rightarrow$ VPSlash \\
& \emph{hon åt redan äpplet}, \emph{hon är redan här}\\
% &\emph{("hon åt äpplet redan")} \\

\textbf{PPartAP} & V2 $\rightarrow$ AP \\
& \emph{det är skrivet},\emph{den skrivna artikeln}
\end{tabular}\\
  
Easy sentences with known words (old testsuite) : 92 out of 118 (78\%) \\
%list of functions
%examples
%difficulties
%sådan
\end{frame}


%kolla upp dessa i stort lexikon
\begin{frame}
\frametitle{Perfect Participle}
\framesubtitle{What type of functions do we need?} 
PPartAP cannot handle \emph{det är givet till henne} or \emph{gågna åren} \\
VPSlash, V2, V?\\
Is the word an adjective or verb? \\
%attribut- predikativ
\begin{tabular}{ll}
Some examples: & \\
\textbf{V} & \\
& \emph{försvunnen}, \emph{slocknad} \\
& \emph{i en gången tids svenska bondesamhälle}  \\% 4872    
\end{tabular} \\
\textbf{V2 or V?}
\emph{en ökad effektiv information} \\
\emph{en minskad benägenhet} \\
\textbf{V2}
\emph{de stegrade hyrorna} \\
\emph{fyra representerade kyrkor} \\
\emph{som aldrig är skalad (industriskalad, skalkokt)} \\
\emph{eftertraktade lägenheter} \\
\textbf{V3}
\emph{skickad till mig}
\textbf{Adj?}
\emph{en fritt vuxen natur } \\
\emph{centralt belägna} \\
\emph{misstänkt } \\% (av)}
\emph{en kvalificerad majoritet} \\
\emph{kokt potatis} \\% (av)}
?
\emph{vår behandlade potatis} \\
\emph{är oroad} \\
\emph{beprövad}  \\ % (av)
\emph{de levererar den doppad i askorbinsyra} \\
\end{frame}


\begin{frame}[containsverbatim]
\frametitle{ReflGenVP}
\framesubtitle{Problem} 
\begin{tabular}{ll}
Adjectives are ok: &  \emph{sin gula lilla katt} \\
% fixa indentering
& \small{\verb|ReflCN NumSg (AdjCN (PositA yellow\_A)|} \\
&\small{\verb|(AdjCN (PositA small\_A) (UseN cat\_N)))|}\\
\end{tabular} \\
\begin{tabular}{ll}
But predeterminers are not: & \emph{alla sina barn} \\
 & \emph{all vår undervisning} \\
 & \emph{sin urgamla rätt att ...}\\  % N2V?
\end{tabular} \\
%hur löses det för vår? (inget problem för det hör ej ihop med subj)

\end{frame}

\begin{frame}
\frametitle{The Grammar}
\framesubtitle{Future work} 
Reciprocals pronouns - \emph{varandra} \\
Add category of words that can be used either as adjectives or determiners - \\
   \emph{fler katter är där}, \emph{fler är där}, \emph{kattern är fler}\\ %sådana
Category used as predeterminer or adverbs - \\
  \emph{bara barn}, \emph{hon bara log} \\
\emph{en sorts katt}, \emph{katter av en sort}
den yngsta är gulast
idioms
minst, lika
"Flickan som såg katten är här"
\textbf{Cleft sentences} är ok!!!
"Det är till uppträdande i denna miljön som utbildningen närmast syftar"
"Det är då som det träder i funktion"
\end{frame}
\begin{frame}
\frametitle{Goals}
\framesubtitle{Goals for christmas} 
\begin{itemize}
\item{Finish and evaluate mapping}
\item{Extend grammar}
    \item{New mapping of saldo - making it easy to keep ut-to-date}
\item{Evaluation}
\end{itemize}
\end{frame}

\begin{frame}
    \frametitle{The end}
\huge\begin{center}Thanks for listening\end{center}
\large\begin{center}Questions?\end{center}
\end{frame}
\end{document}

